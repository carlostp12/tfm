	Finalizada la fase de inicialización del servidor, en la pantalla que se nos muestra, aceptamos las advertencias y presionamos el botón Ver Certificados (Fig. \ref{caroot_fig14}), aceptamos la advertencia y si pulsamos sobre el desplegable, nos aparece una lista con los certificados de acceso que tenemos instalados, debiendo seleccionar el certificado que utilizaremos para autenticarnos, presionando posteriormente el botón Autenticar. En este punto el único certificado que tendremos disponible es el de operador, ya que acabamos de finalizar el proceso de inicialización.

	\begin{figure}[H]
		\centering
		\includegraphics[scale=.5]{./img/caroot_fig14.png}
		\caption{Pantalla de selección de certificados de interfaz de administración.}
		\label{caroot_fig14}
	\end{figure}

	En caso de tener asignado más de un rol para dicho usuario, el siguiente paso es la selección del rol con el que queremos acceder (Fig. \ref{caroot_fig15}), para ello presionamos el desplegable y nos aparecen los diferentes roles que tenemos disponibles para el certificado seleccionado (Fig. \ref{caroot_fig16}), seleccionamos aquel con el que queremos acceder y presionamos sobre el botón Continuar. En caso de tener asignado un único rol como es el caso de acabar de inicializar, se omite este paso.

	\begin{figure}[H]
		\centering
		\includegraphics[scale=.5]{./img/caroot_fig15.png}
		\caption{Pantalla de selección de rol de interfaz de administración.}
		\label{caroot_fig15}
	\end{figure}

	\begin{figure}[H]
		\centering
		\includegraphics[scale=.5]{./img/caroot_fig16.png}
		\caption{Pantalla con los roles disponibles para el certificado instalado.}
		\label{caroot_fig16}
	\end{figure}