	\subparagraph{Mantenimiento base de datos}\label{mantenimiento_bbdd}

	Mediante esta tarea, podemos consultar el nivel de fragmentación de la base de datos, así como realizar una compactación ligera (la cual no bloquea la base de datos) y una compactación completa (la cual bloquea la base de datos). Además, podremos configurar un backup periódico, debiendo poner para ello cada cuantas horas queremos que éste sea generado o bien, poniendo a qué hora del día, queremos que se genere el mismo. (Fig. \ref{ra_32}).  

	\begin{figure}[H]
		\centering
		\includegraphics[scale=.5]{./img/ra_32.png}
		\caption{Pantalla para fragmentar la base de datos.}
		\label{ra_32}
	\end{figure}