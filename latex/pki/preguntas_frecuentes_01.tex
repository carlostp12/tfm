	\section{Preguntas frecuentes}

	\begin{enumerate}
		\item \textbf{Si no nos permite ver el listado de certificados con los que podemos acceder a la interfaz y por lo cual tampoco acceder a ella:}

		Si intenta acceder con Internet Explorer, compruebe que la URL se encuentra en sitios de confianza del navegador.
		\item \textbf{La configuración de la seguridad de Java, depende de la versión del mismo.}
		\ifthenelse{\equal{\jobname}{\detokenize{ra}}}
		{
			\item \textbf{Si se actualizan los certificados de miniCA o de servidor web, en la CA, hay que volver a configurarlo en la RA (Ver \ref{ra_config_ca}) para poder establecer la conexión con la CA.}

			\item \textbf{Si al emitir certificados o sincronizar políticas da error de no haberse podido establecer conexión con la CA:}

			Comprobar que está configurado el certificado de CA Raíz y que está configurada la CA, para llevar a cabo dichas funciones, debemos ver las secciones \ref{ra_importar_ca_cert} y \ref{ra_config_ca} respectivamente.
		}{}
		
		\item \textbf{Si intentamos acceder desde Google Chrome y el CN del certificado no coincide con IP\_servidor, el navegador mostrará como que el certificado no es seguro, por lo que tendremos que presionar `acceder de todas formas'.}
		
		
		\item \textbf{Al intentar generar o restaurar un backup de datos aparece un error 504, o Timeout:}

		Puede ocurrir que al intentar generar un backup de datos, dependiendo del tamaño de la base de datos, la petición tarde tanto en ejecutarse que aparezca un error debido a la configuración servidor, de la red o del navegador cliente. Aunque aparezca este error, el servicio de backup seguirá ejecutándose hasta terminar de generar los archivos correspondientes. Esto no es ningún problema simplemente volvemos a acceder y pasado un tiempo suficiente para que finalice el proceso (tiempo a determinar dependiendo del tamaño de los datos) tendremos los archivos disponibles para descargar.

		En el caso de la restauración, la solución del problema es similar, con la particularidad de que puede ocurrir también en la subida de los archivos de datos, simplemente, volvemos a acceder para comprobar ha terminado el proceso de subida de archivos o en su caso, del proceso de ejecución de la restauración del backup.
		\ifthenelse{\equal{\jobname}{\detokenize{tsa}}}
		{
			\item \textbf{El servicio TSP no funciona. Al intentar firmar un documento, aparece un mensaje de error o la URL del TSP está inaccesible, a pesar de que la aplicación está accesible:}

			Compruebe con el administrador de seguridad que 
			\begin{enumerate}
				\item El certificado raíz de TSA ha sido generado y está firmado por la CA correspondiente (ver \ref{config_tsa_certs} en este documento).
				\item Compruebe que el certificado raíz tiene el uso extendido de clave `timestamping'.
				\item Revise la configuración de TSA (ver \ref{config_tsa_stamp}).
			\end{enumerate}
		}{}
		
	\end{enumerate}