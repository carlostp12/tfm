	\subsubsection{Administración web para el rol Operador de Sistemas}

	Mediante el rol operador de sistema, se autoriza para generar y configurar CRLs, monitorizar el hardware y sistema operativo, administrar servicios, copia de seguridad de datos y ver los logs de software, sistema operativo y base de datos.

	La pantalla de administración web para el rol operador de sistema es (Fig. \ref{caroot_fig126}).

	\begin{figure}[H]
		\centering
		\includegraphics[scale=.5]{./img/caroot_fig126.png}
		\caption{Pantalla de inicio de administración web para el rol operador de sistema.}
		\label{caroot_fig126}
	\end{figure}

	\paragraph{Operaciones}
	\ifthenelse{\equal{\jobname}{\detokenize{ca_sub}} \OR \equal{\jobname}{\detokenize{ca_root}}}
	{
		\subparagraph{Generación de CRL}

		Para poder llevar a cabo esta operación, debe estar configurada la CRL, operación que debe ser llevada a cabo por el administrador de seguridad. 

		En esta sección, el administrador de sistemas puede generar CRL`s (Fig. \ref{caroot_fig127}), para ello se presiona el botón \textit{Generar}, el sistema nos muestra en la ventana la información de la CRL (Fig. \ref{caroot_fig128}) y la cual podremos descargarnos.

		\begin{figure}[H]
			\centering
			\includegraphics[scale=.5]{./img/caroot_fig127.png}
			\caption{Pantalla para generar CRL.}
			\label{caroot_fig127}
		\end{figure}

		\begin{figure}[H]
			\centering
			\includegraphics[scale=.5]{./img/caroot_fig128.png}
			\caption{Pantalla con la información del CRL.}
			\label{caroot_fig128}
		\end{figure}
	}{}

	\subparagraph{Monitorización}

	\begin{itemize}
		\item \textbf{Hardware}
	\end{itemize}

	En este apartado, el sistema nos permite ver la monitorización del hardware actualizada (Fig. \ref{caroot_fig129}).

	\begin{figure}[H]
		\centering
		\includegraphics[scale=.5]{./img/caroot_fig129.png}
		\caption{Pantalla con la monitorización del Hardware.}
		\label{caroot_fig129}
	\end{figure}

	\begin{itemize}
		\item \textbf{Sistema Operativo}
	\end{itemize}

	En esta sección, el sistema nos muestra la monitorización del sistema operativo, también podemos escribir el identificador de proceso y darle a terminar, para de este modo finalizar el proceso indicado (Fig. \ref{caroot_fig130}).

	\begin{figure}[H]
		\centering
		\includegraphics[scale=.5]{./img/caroot_fig130.png}
		\caption{Pantalla con la monitorización del Sistema Operativo.}
		\label{caroot_fig130}
	\end{figure}

	\subparagraph{Administración de servicios}

	Mediante esta opción, el usuario puede realizar las siguientes acciones:

	\begin{itemize}
		\item Reiniciar el servidor web
		\item Parar el servicio de la base de datos
		\item Iniciar el servicio de la base de datos
		\item Reiniciar el servicio de la base de datos
		\item Reiniciar el servicio https.
	\end{itemize}

	Para ello, debemos presionar sobre el icono correspondiente a la acción que queramos llevar a cabo (Fig. \ref{caroot_fig131}).

	\begin{figure}[H]
		\centering
		\includegraphics[scale=.5]{./img/caroot_fig131.png}
		\caption{Pantalla con los servicios que podemos administrar.}
		\label{caroot_fig131}
	\end{figure}

	\subparagraph{Copia de seguridad de datos}

	La finalidad de esta funcionalidad es la de generar y descargar el fichero de backups de datos, para ello debemos presionar el botón \textit{Generar}, de forma que se actualice el backup de datos y posteriormente presionamos sobre \textit{Descargar} para guardarnos el backup de datos y poderlo importar posteriormente.

	La primera vez que accedemos a este punto de la interfaz, si no hay un backup automático ya generado (ver apartado \ref{mantenimiento_bbdd}) , podemos apreciar la siguiente pantalla(Fig. \ref{caroot_fig132}).

	\begin{figure}[H]
		\centering
		\includegraphics[scale=.5]{./img/caroot_fig132.png}
		\caption{Pantalla de backup de datos.}
		\label{caroot_fig132}
	\end{figure}

	En este caso, no hay ningún backup de datos generado.

	Una vez pulsemos en `Generar' se inicia el proceso de generación de backup. Cuando este proceso termine, nos mostrará en esta misma pantalla los archivos que componen el backup de datos para que podamos descargarlos  (Fig. \ref{caroot_fig133}). Después podremos restaurar dicho backup con el usuario con el rol correspondiente (Ver apartado \ref{restauracion_datos}).

	\begin{figure}[H]
		\centering
		\includegraphics[scale=1]{./img/caroot_fig133.png}
		\caption{Pantalla para generar y descargar backup de datos.}
		\label{caroot_fig133}
	\end{figure}

	Observar que el backup se compone de dos partes:

	\begin{itemize}
		\item Un archivo .zip descriptor del back up y de los archivos que lo componen.
		\item Uno o más archivos (el número depende del tamaño de nuestra base de datos) que componen los datos de la base de datos.
	\end{itemize}

	Cuando se procede a ejecutar un proceso de un backup manual de datos, las siguientes operaciones no están disponibles en la interfaz:

	\begin{enumerate}
		\item Otro backup manual de datos.
		\item Restauración.
		\item Backup manual de claves.
		\item Parar, iniciar o reiniciar los servicios de base de datos.
	\end{enumerate}

	Hasta que nuestra operación de generación de backup haya terminado.

	\paragraph{Consultas}
	\ifthenelse{\equal{\jobname}{\detokenize{ca_sub}} \OR \equal{\jobname}{\detokenize{ca_root}} \OR \equal{\jobname}{\detokenize{ra}}}
	{
		\subparagraph{CRL`s}

		En este apartado, podemos llevar a cabo la búsqueda de CRL`s, pudiendo filtrar para ello por los campos del formulario que podemos ver en (Fig. \ref{caroot_fig134}), si no filtramos por ningún campo, muestra un listado con todas las CRL`s, debiendo presionar el botón \textit{Buscar}, si por el contrario presionamos el botón \textit{Limpiar} búsqueda, los campos del formulario que hubiésemos completado, se restablecerán quedando con el estado inicial.

		Si presionamos sobre el \textit{ID}, podemos descargarnos la CRL, y si presionamos sobre \textit{Cuerpo} de CRL, nos muestra la CRL.

		\begin{figure}[H]
			\centering
			\includegraphics[scale=.5]{./img/caroot_fig134.png}
			\caption{Pantalla para buscar CRLS y el listado de las mismas.}
			\label{caroot_fig134}
		\end{figure}
	}{}

	\subparagraph{Logs}

	Mediante esta operación, podemos descargarnos los logs de software de la interfaz, los logs de servidor web y los logs de base de datos (Fig. \ref{caroot_fig135}). Si marcamos el campo \textit{Seleccionar} y presionamos \textit{Eliminar}, una vez aceptada la advertencia, se borra el log.

	\begin{figure}[H]
		\centering
		\includegraphics[scale=.5]{./img/caroot_fig135.png}
		\caption{Pantalla de Logs.}
		\label{caroot_fig135}
	\end{figure}