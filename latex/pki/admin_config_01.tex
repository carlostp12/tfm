	\subsubsection{Administración web para el rol Administrador de Sistemas}

	\ifthenelse{\equal{\jobname}{\detokenize{ca_root}}}{
		Cuando intentamos acceder a este rol, en la pantalla donde se nos muestra la información de la última conexión, podemos actualizar la hora y fecha del equipo, para ello presionamos sobre Actualizar (Fig. \ref{caroot_fig74}).

		\begin{figure}[H]
			\centering
			\includegraphics[scale=.5]{./img/caroot_fig74.png}
			\caption{Pantalla de información de última conexión para el Administrador de Sistemas desde donde podemos poner en hora.}
			\label{caroot_fig74}
		\end{figure}

	}{}
	
	Mediante este rol, se autoriza para configurar los parámetros de red, actualizar software, actualizar firmware, así como el mantenimiento de la base de datos\ifthenelse{\equal{\jobname}{\detokenize{ra}}}{, además de configurar la CA y el acceso a LDAP.}{\ifthenelse{\equal{\jobname}{\detokenize{va}}}{ o los datos de acceso a la CA.}{.}}

	La pantalla de administración web para el rol administrador de sistema es (Fig. \ref{caroot_fig75}).

	\begin{figure}[H]
		\centering
		\includegraphics[scale=.5]{./img/caroot_fig75.png}
		\caption{Pantalla de inicio de administración web para el rol administrador de sistema.}
		\label{caroot_fig75}
	\end{figure}

	\paragraph{Configuración}

	\ifthenelse{\equal{\jobname}{\detokenize{ca_root}}}{}
	{
		\subparagraph{Configuración de red}
		Podemos configurar el servicio NTP, para ello debemos escribir los servidores separados con comas en el hueco destinado para ello. También podemos hacer una prueba del servicio NTP. Vemos todo ello en la Fig. \ref{caroot_fig76}.

		\begin{figure}[H]
			\centering
			\includegraphics[scale=.5]{./img/casub_5.png}
			\caption{Pantalla configuración de red.}
			\label{caroot_fig76}
		\end{figure}
	}

	\subparagraph{Actualización de Software}

	En esta sección, podemos actualizar la plataforma (Software y Firmware), debiendo para ello presionar el botón Examinar, seleccionar el certificado del operador que va a proporcionar la actualización (este será proporcionado por Realsec) y presionar el botón Actualizar. Posteriormente se presiona Examinar, y se selecciona el paquete correspondiente a la actualización, el cual va a estar cifrado y firmado, y presionar el botón Actualizar (Fig. \ref{caroot_fig78}).

	\begin{figure}[H]
		\centering
		\includegraphics[scale=1]{./img/caroot_fig78.png}
		\caption{Pantalla para actualizar la plataforma.}
		\label{caroot_fig78}
	\end{figure}
	
	\subparagraph{Archivar datos}
	
	En esta sección, podemos archivar datos (Certificados, Logs y CRL) y moverlos a una base de datos diferente. Para realizar el proceso se debe proporcionar la siguiente información (Fig. \ref{caroot_archivate_fig1}):
	\begin{itemize}
		\item \textbf{Host}: Debe indicar la dirección IP del servidor de base de datos donde se almacenarán los datos a archivar.
		\item \textbf{Port}: Debe indicar el puerto del servidor de base de datos donde se almacenarán los datos a archivar.
		\item \textbf{Usuario}: Debe indicar el usuario del servidor de base de datos donde se almacenarán los datos a archivar.
		\item \textbf{Contraseña}: Debe indicar la contraseña del usuario del servidor de base de datos donde se almacenarán los datos a archivar.
		\item \textbf{Nombre BBDD}: Debe indicar el nombre de la base de datos en el servidor de base de datos donde se almacenarán los datos a archivar.
		\item \textbf{Datos a archivar}: Debe seleccionar los datos archivar.
		\item \textbf{Fecha desde-hasta}: Debe indicar el rango de fechas de los datos que desea archivar.
		\item \textbf{Incluir CREATE TABLE}: Debe marcar esta opción solo si la base de datos NO contiene todavía las tablas donde se almacenarán los datos.
	\end{itemize}
	
	\begin{figure}[H]
		\centering
		\includegraphics[scale=.5]{./img/archivar_1.png}
		\caption{Pantalla para archivar datos.}
		\label{caroot_archivate_fig1}
	\end{figure}
	
	Una vez introducidos los datos, al presionar el botón \textit{Archivar datos} podremos ver el resultado de la operación (Fig. \ref{caroot_archivate_fig2}).
	
	\begin{figure}[H]
		\centering
		\includegraphics[scale=.5]{./img/archivar_2.png}
		\caption{Pantalla con resultado de archivar datos.}
		\label{caroot_archivate_fig2}
	\end{figure}
	