\pagenumbering{roman}
\setcounter{page}{1}
\pagestyle{plain}

%%%%%%%%%%%%%%%%
%%% CREDITS %%%
%%%%%%%%%%%%%%%%
\chapter*{Credits/Copyright}

A page with the specification of credits/copyright for the project (either application on one side and documentation on the other, or unified), as well as the use of third-party trademarks, products or services (including source code). If a person other than the author collaborated on the project, their identity and what they did must be explicitly stated.

Below is the most common case, but it can be modified for any other alternative:

\vspace{1cm}

\begin{figure}[ht]
\centering
\includegraphics[scale=1]{images/license.png}
\end{figure}

Copyright (c) 2025, Carlos Toro Peñas.
Attribution-NonCommercial-NoDerivs 3.0 Spain (CC BY-NC-ND 3.0 ES) 
\href{https://creativecommons.org/licenses/by-nc-nd/3.0/es/}{3.0 Spain of CreativeCommons}.

%%%%%%%%%%%%%
%%% RECORD %%%
%%%%%%%%%%%%%
\chapter*{FINAL PROJECT RECORD}

\begin{table}[ht]
\centering{}
\renewcommand{\arraystretch}{2}
\begin{tabular}{r | l}
\hline
Title of the project: & \titulo \\
\hline
Author's name: & \autor \\
\hline
Collaborating teacher's name: & \supervisor \\
\hline
PRA's name: & First and last name\\
\hline
Delivery date (mm/yyyy): & MM/YYYY\\
\hline
Degree or program: & Master's degree in Data Sicience\\
\hline
Final Project area: & 4\\
\hline
Language of the project: & English\\
\hline
Keywords: & \keywords\\
\hline
\end{tabular}
\end{table}

%%%%%%%%%%%%%%%%%%%
%%% DEDICATION %%%
%%%%%%%%%%%%%%%%%%%
\chapter*{Dedication/Quote}

To my wife, to whom this work owes more than she imagines.

%%%%%%%%%%%%%%%%%%%
%%% Acknowledgements %%%
%%%%%%%%%%%%%%%%%%%
%\chapter*{Acknowledgements}
%
%If deemed appropriate, mention the people, companies or institutions that have contributed to the realization of this project.

%%%%%%%%%%%%%%%%
%%% ABSTRACT %%%
%%%%%%%%%%%%%%%%
\chapter*{Abstract}
\addcontentsline{toc}{chapter}{Abstract}

\onehalfspacing

This work focuses primary on the application of density-based algorithms to datasets obtained from various surveys, such as the Two-degree Field Galaxy Redshift Survey (2dFGRS) and the Sloan Digital Sky Survey (SDSS). As a result of this application, an hyperparameters adjutment and an assessment the performance will be conducted to identify the strengths and weaknesses of these algorithms in actual galactic groups detection. In the future, these algorithms may be applied to new surveys and other regions of the sky.

Text with a summary of the project, that is, a concise explanation of the project/problem addressed, its objectives/resolution methods, and the results and conclusions (it cannot be a list, but rather a continuous text written in a structured way). If a reference is necessary in this text, it will be noted at the bottom of the same page. In this section, a more literary and colloquial language can be used than for the rest of the document.

The Abstract will be written twice. One version must be \textbf{obligatorily in English}. The other version must be written in Catalan or Spanish. If the rest of the document is not written in English, it will be necessary to write the second version of the Abstract in the language used for the rest of the report. The word Abstract will be changed to \textbf{Resum}'' or \textbf{Resumen}'' in the Catalan and Spanish versions, respectively.

Recommended length: maximum 250 words.

How to write a good Abstract:

\href{http://www.ece.cmu.edu/~koopman/essays/abstract.html}{http://www.ece.cmu.edu/~koopman/essays/abstract.html}

\vspace{1.5cm}


\textbf{Keywords}: \keywords.

%%%%%%%%%%%%%%%%
%%% RESUMEN %%%
%%%%%%%%%%%%%%%%
\chapter*{Resumen}
\addcontentsline{toc}{chapter}{Resumen}

\onehalfspacing

Este trabajo tiene como tema central la aplicación de diferentes algoritmos basados en densidad a juegos de datos obtenidos de estudios como (Two-degree Field Galaxy Redshift Survey 2DFGRS) y Sloan Digital Sky Survey (SDSS). Como resultado de esa aplicación, se hará un ajuste de sus hiperparámetros así como una evaluación del desempeño de tales algoritmos, sus fortalezas y debilidades en su habilidad para la detección de cúmulos galácticos. Futuramente, se podrán realizar aplicaciones de estos algoritmos a nuevos estudios y otras regiones del cielo.

\vspace{1.5cm}


\textbf{Palabras clave}: \keywordses.