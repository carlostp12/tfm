\pagenumbering{roman}
\setcounter{page}{1}
\pagestyle{plain}

%%%%%%%%%%%%%%%%
%%% CREDITS %%%
%%%%%%%%%%%%%%%%
\chapter*{Credits/Copyright}

A page with the specification of credits/copyright for the project (either application on one side and documentation on the other, or unified), as well as the use of third-party trademarks, products or services (including source code). If a person other than the author collaborated on the project, their identity and what they did must be explicitly stated.

Below is the most common case, but it can be modified for any other alternative:

\vspace{1cm}

\begin{figure}[ht]
\centering
\includegraphics[scale=1]{images/license.png}
\end{figure}

Attribution-NonCommercial-NoDerivs 3.0 Spain (CC BY-NC-ND 3.0 ES) 

\href{https://creativecommons.org/licenses/by-nc-nd/3.0/es/}{3.0 Spain of CreativeCommons}.

%%%%%%%%%%%%%
%%% RECORD %%%
%%%%%%%%%%%%%
\chapter*{FINAL PROJECT RECORD}

\begin{table}[ht]
\centering{}
\renewcommand{\arraystretch}{2}
\begin{tabular}{r | l}
\hline
Title of the project: & Descriptive of the project\\
\hline
Author's name: & First and last name\\
\hline
Collaborating teacher's name: & First and last name\\
\hline
PRA's name: & First and last name\\
\hline
Delivery date (mm/yyyy): & MM/YYYY\\
\hline
Degree or program: & Curriculum\\
\hline
Final Project area: & Name of the TF course\\
\hline
Language of the project: & Catalan, Spanish, or English\\
\hline
Keywords & Maximum 3 keywords\\
\hline
\end{tabular}
\end{table}

%%%%%%%%%%%%%%%%%%%
%%% DEDICATION %%%
%%%%%%%%%%%%%%%%%%%
\chapter*{Dedication/Quote}

Brief words of dedication and/or a quote.

%%%%%%%%%%%%%%%%%%%
%%% Acknowledgements %%%
%%%%%%%%%%%%%%%%%%%
\chapter*{Acknowledgements}

If deemed appropriate, mention the people, companies or institutions that have contributed to the realization of this project.

%%%%%%%%%%%%%%%%
%%% ABSTRACT %%%
%%%%%%%%%%%%%%%%
\chapter*{Abstract}
\addcontentsline{toc}{chapter}{Abstract}

\onehalfspacing

Text with a summary of the project, that is, a concise explanation of the project/problem addressed, its objectives/resolution methods, and the results and conclusions (it cannot be a list, but rather a continuous text written in a structured way). If a reference is necessary in this text, it will be noted at the bottom of the same page. In this section, a more literary and colloquial language can be used than for the rest of the document.

The Abstract will be written twice. One version must be \textbf{obligatorily in English}. The other version must be written in Catalan or Spanish. If the rest of the document is not written in English, it will be necessary to write the second version of the Abstract in the language used for the rest of the report. The word Abstract will be changed to \textbf{Resum}'' or \textbf{Resumen}'' in the Catalan and Spanish versions, respectively.

Recommended length: maximum 250 words.

How to write a good Abstract:

\href{http://www.ece.cmu.edu/~koopman/essays/abstract.html}{http://www.ece.cmu.edu/~koopman/essays/abstract.html}

\vspace{1.5cm}

\textbf{Keywords}: Keywords related to the project separated by commas. For example, for this document, they could be Model, Guideline, Template, Report, Bachelor's/Master's Thesis.
