\phantomsection
\pagestyle{fancy}

\chapter{Introduction}
\onehalfspacing
%\addcontentsline{toc}{chapter}{Introduction}
%\section{Introduction}
\section{Justification of interest and relevance}

In recent times, it has been established that the large-scale structure of the universe is formed by a vast cosmic web \cite{Eniasto:2014} primarily composed of dark matter. The topology of this cosmic web consists of filaments that enclose large voids \cite{Anatole:2024}. These filaments, made up mainly by dark matter halos which embed the baryonic matter, mainly comprised of galaxy clusters and interstellar matter. At the filaments of dark matter and mainly at the intersection points of them is where the largest and most populated clusters and superclusters of galaxies reside, thus forming the largest structures of the visible universe.

Determining the structure of these clusters is therefore crucial for understanding the matter distribution in the universe. This is where clustering algorithms come into play. For reasons that will be discussed further, the density-based algorithms may be the most appropriate for this study.

In this work we want to partially answer how density-based algorithms can be applied to determine how the matter is groupd in the universe.

\section{Personal motivation}

Given my background and strong interest in astrophysics and cosmology, upon entering the field of Data Science, it is easy to recognize the vast potential for applying the multiple Machine Learning (ML) techniques to these scientific domains. In particular, the study of the large-scale structure of the Universe is, without a doubt, one of the most fascinating topics in science today.

\begin{figure}[h]
\centering
\includegraphics[width=0.5\textwidth]{./figs/M33_2.jpg}
\caption{M33: A galaxy belonging to our local group.}
\label{fig:sample_figure}
\end{figure}


\section{Goals definition}

There is a list of objectives I aim to reach with the present work:
\begin{itemize}
	\item Generate a visualization map of the data object in this study.
	\item Apply some density-based algorithm to galaxy and galaxy-groups datasets obtained from SDSS and 2DFGRS.
	\item Figure out which of these algorithms work better and determine the possible causes.
	\item Create a validation methods to obtain a hyperparameters adjustements that optimize the gruops detection.
	\item Detect possible methods to improve this study.
\end{itemize}

It is also expected to achieve some dregree of approximation to actual groups of galaxies through the clusters obtained by the density-based methods.

% sustentability from now it will be leave out
% \subsection{Sustainability, diversity, and ethical/social challenges}

This section should assess the positive/negative impact of the project in the following dimensions. It is not required to reach a positive impact in any/all dimensions, but it is necessary to consider and discuss whether there is an impact or not from the beginning of the project.

\begin{description}
    \item[Sustainability] In the development of the project or during its entire lifecycle (e.g., deployment, retirement), does the output of this project have an impact on sustainability and/or ecological footprint (energy/resource consumption/savings, waste, pollution, depletion of raw materials)? Is it affected by laws or regulations on this matter? Considering another perspective, does it affect any of the Sustainable Development Goals (SDG) related to these dimensions? If it does not have any impact, either positive or negative, you should explain how you reached this conclusion and justify your answer.
    \item[Ethical behaviour and social responsibility] Is the outcome of the project too technical to have any positive/negative impact in ethical/social aspects? Does it have an impact on laws/regulations (data, privacy, labour, intellectual property, personal security, …)? Does it adhere to the deontological principles of the profession? Does it endanger/improve/worsen any job position? If it does not have any impact, either positive or negative, you should explain how you reached this conclusion and justify your answer.
    \item[Diversity, gender and human rights] Is the result of this project so technical that it has no positive/negative impact in terms of gender, diversity, or human rights? And in any laws/regulations? And in terms of accessibility, disability, ergonomics and/or information security? If it does not have any impact, either positive or negative, you should explain how you reached this conclusion and justify your answer.
\end{description}


\section{Methodology and project development}

 Unsupervised algorithms and in particular density-based will be applied in to datasets sampled from surveys such as the SDSS and 2DFGRS to generate models for galaxy clustering.

To evaluate the performance of these models, the following criteria will be evaluate:

\begin{itemize}
    \item {Detected Clusters}: Groups successfully classified as clusters (often referred to as True Positives at the group level).

	\item {Undetected Clusters}: Groups not found or not identified in the clusters set (equivalent to True Negatives at the group level).

	\item {Cluster Purity Ratio}: The proportion of members in a detected cluster that truly belong to the underlying group/structure.

	\item {Cluster Completeness Ratio}: The proportion of members of a true underlying group/structure that are successfully included in the detected cluster.

	\item {Misclassified Members}: Individual data points (galaxies) belonging to a true group but classified outside of any detected cluster. (Often referred to as False Negatives at the individual member level).

	\item {External Data Classified as Members}: Individual data points (galaxies) not belonging to a true group but erroneously classified inside a detected cluster. (Often referred to as False Positives at the individual member level).
\end{itemize}

For this study, the environment choosen is RStudio to deploy scritps in R programming language.

\section{Schedule}

A Gantt diagram has been created in the \ref{fig:stages_figure} to show the different stages of the project development. The stages have been grouped on three sets:

\begin{figure}[h]
\centering
\includegraphics[width=1.0\textwidth]{./figs/gantt_conogram.png}
\caption{Stages of the project.}
\label{fig:stages_figure}
\end{figure}

\begin{itemize}
	\item In green: planning: gethering resources and objectives of the project.
	\item In red: technical development of the project: design, data pocessing, methods application and assesing the outcomes.
	\item In blue: reasearch and writing the outcomes.
\end{itemize}	
In most of the time there is a ovelapping of stages, this is due following:
\begin{itemize}
	\item Inital stages: starting the project are composed of several tasks.
	\item There are two different stages of objectives, one on the starting of the project and other one in the implementation.
\end{itemize}
