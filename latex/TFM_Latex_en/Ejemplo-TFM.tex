\documentclass[12pt,a4paper,twoside]{book}
\usepackage{graphicx}
\usepackage{setspace} %double spacing for text, single for captions, footnotes, etc.
%\usepackage{hypernat} %substitute for cite that allows hyperlinks
\usepackage{natbib} % substitute for 'hypernat' that works on Windows.
\usepackage[english]{babel}
\usepackage[utf8]{inputenc}
\usepackage{color}
\usepackage{hhline} % extended styles for tables
\usepackage{multirow}
\usepackage{subfigure}
\usepackage{acronym}
\usepackage{hyperref}
\usepackage{amsmath,amsmath,amssymb}
\usepackage{fancyhdr}
\usepackage{epsfig, amsmath}
\usepackage{algorithm}
\usepackage{algorithmic}

% general settings
\hypersetup{
linktocpage=true,
colorlinks=true,
linkcolor=blue,
citecolor=blue,
}
\definecolor{Hgray}{gray}{0.6}

\newenvironment{definition}[1][Definition]{\begin{trivlist}
\item[\hskip \labelsep {\bfseries #1}]}{\end{trivlist}}

\setlength{\topmargin}{0cm}
\setlength{\textheight}{23cm}
\setlength{\textwidth}{17cm}
\setlength{\oddsidemargin}{0cm}
\setlength{\evensidemargin}{0cm}
\setlength{\headheight}{1cm}

% indicates that 'sub-sub-sections' are numbered and appear in the index
\setcounter{secnumdepth}{3}
\setcounter{tocdepth}{2}

% settings for code
\renewcommand{\algorithmicrequire}{\textbf{Input:}}
\renewcommand{\algorithmicensure}{\textbf{Output:}}

%%%%%%%%%%%%
% DOCUMENT %
%%%%%%%%%%%%
\begin{document}

\setcounter{section}{0} % Resets the section counter to 0 at the document's start
\renewcommand{\thesection}{\arabic{section}} % Changes the section numbering scheme

%%%%%%%%%%%%%%%%%%%%%%%%%%%%%%%%%%%%%
% Definning new variables			%
%%%%%%%%%%%%%%%%%%%%%%%%%%%%%%%%%%%%%
\newcommand{\titulo}{Density-based algorithms applied to galaxy groups catalogs}
\newcommand{\subtitulo}{-}
\newcommand{\autor}{Carlos Toro Peñas}
\newcommand{\supervisor}{Laura Ruíz Dern}
\newcommand{\profesor}{-}
\newcommand{\keywords}{Clustering, Galaxy groups, cosmology}
\newcommand{\keywordses}{Clusterización, cúmulos de galaxias, cosmología}

% cover page
\newpage
\thispagestyle{empty}

\baselineskip 2em

%\vspace*{1cm}

\centerline{\includegraphics[width=0.6\textwidth]{images/UOC-logo}}
\begin{center}
\textsc{Open University of Catalonia (UOC) \
Master's Degree in Data Science \
}

%\centerline {\pic{UOC}{4cm}}

\vspace*{1.5cm}

\textsc{\Large MASTER'S THESIS}

\vspace*{0.5cm}

\textsc{\large Area: YYY}

%\textbf{\Huge VirtualTechLab Model: }

\vspace*{2.0cm}

\textbf{\Large xxx thesis title xxx}

\textbf{\large xxx subtitle (if any) xxx}

\vspace{2.5cm}
\baselineskip 1em

\baselineskip 2em
-----------------------------------------------------------------------------\\
Author: Full name of the student\\
Tutor: Name of the teaching collaborator\\
Professor: Name of the professor responsible for the thesis area\\
-----------------------------------------------------------------------------\\
\vspace*{1.5cm}
Barcelona, \today

\end{center}

\newpage
\pagestyle{empty}
\hfill
\newpage

% abstract
\pagenumbering{roman}
\setcounter{page}{1}
\pagestyle{plain}

%%%%%%%%%%%%%%%%
%%% CREDITS %%%
%%%%%%%%%%%%%%%%
\chapter*{Credits/Copyright}

A page with the specification of credits/copyright for the project (either application on one side and documentation on the other, or unified), as well as the use of third-party trademarks, products or services (including source code). If a person other than the author collaborated on the project, their identity and what they did must be explicitly stated.

Below is the most common case, but it can be modified for any other alternative:

\vspace{1cm}

\begin{figure}[ht]
\centering
\includegraphics[scale=1]{images/license.png}
\end{figure}

Attribution-NonCommercial-NoDerivs 3.0 Spain (CC BY-NC-ND 3.0 ES) 

\href{https://creativecommons.org/licenses/by-nc-nd/3.0/es/}{3.0 Spain of CreativeCommons}.

%%%%%%%%%%%%%
%%% RECORD %%%
%%%%%%%%%%%%%
\chapter*{FINAL PROJECT RECORD}

\begin{table}[ht]
\centering{}
\renewcommand{\arraystretch}{2}
\begin{tabular}{r | l}
\hline
Title of the project: & Descriptive of the project\\
\hline
Author's name: & First and last name\\
\hline
Collaborating teacher's name: & First and last name\\
\hline
PRA's name: & First and last name\\
\hline
Delivery date (mm/yyyy): & MM/YYYY\\
\hline
Degree or program: & Curriculum\\
\hline
Final Project area: & Name of the TF course\\
\hline
Language of the project: & Catalan, Spanish, or English\\
\hline
Keywords & Maximum 3 keywords\\
\hline
\end{tabular}
\end{table}

%%%%%%%%%%%%%%%%%%%
%%% DEDICATION %%%
%%%%%%%%%%%%%%%%%%%
\chapter*{Dedication/Quote}

Brief words of dedication and/or a quote.

%%%%%%%%%%%%%%%%%%%
%%% Acknowledgements %%%
%%%%%%%%%%%%%%%%%%%
\chapter*{Acknowledgements}

If deemed appropriate, mention the people, companies or institutions that have contributed to the realization of this project.

%%%%%%%%%%%%%%%%
%%% ABSTRACT %%%
%%%%%%%%%%%%%%%%
\chapter*{Abstract}
\addcontentsline{toc}{chapter}{Abstract}

\onehalfspacing

Text with a summary of the project, that is, a concise explanation of the project/problem addressed, its objectives/resolution methods, and the results and conclusions (it cannot be a list, but rather a continuous text written in a structured way). If a reference is necessary in this text, it will be noted at the bottom of the same page. In this section, a more literary and colloquial language can be used than for the rest of the document.

The Abstract will be written twice. One version must be \textbf{obligatorily in English}. The other version must be written in Catalan or Spanish. If the rest of the document is not written in English, it will be necessary to write the second version of the Abstract in the language used for the rest of the report. The word Abstract will be changed to \textbf{Resum}'' or \textbf{Resumen}'' in the Catalan and Spanish versions, respectively.

Recommended length: maximum 250 words.

How to write a good Abstract:

\href{http://www.ece.cmu.edu/~koopman/essays/abstract.html}{http://www.ece.cmu.edu/~koopman/essays/abstract.html}

\vspace{1.5cm}

\textbf{Keywords}: Keywords related to the project separated by commas. For example, for this document, they could be Model, Guideline, Template, Report, Bachelor's/Master's Thesis.

\newpage

\pagestyle{fancy}
\renewcommand{\chaptermark}[1]{ \markboth{#1}{}}
\renewcommand{\sectionmark}[1]{\markright{ \thesection.\ #1}}
\lhead[\fancyplain{}{\bfseries\thepage}]{\fancyplain{}{\bfseries\rightmark}}
\rhead[\fancyplain{}{\bfseries\leftmark}]{\fancyplain{}{\bfseries\thepage}}
\cfoot{}

% table of contents
\cleardoublepage
\phantomsection
\addcontentsline{toc}{chapter}{Table of Contents}
\tableofcontents
% list of figures
\cleardoublepage
\phantomsection
\addcontentsline{toc}{chapter}{List of Figures}
\listoffigures

% list of tables
%\cleardoublepage
%\phantomsection
%\addcontentsline{toc}{chapter}{List of Tables}
%\listoftables

\thispagestyle{empty}

\pagenumbering{arabic}

%\pagestyle{fancy}
%\renewcommand{\chaptermark}[1]{ \markboth{#1}{}}
%\renewcommand{\sectionmark}[1]{\markright{ \thesection.\ #1}}
%\lhead[\fancyplain{}{\bfseries\thepage}]{\fancyplain{}{\bfseries\rightmark}}
%\rhead[\fancyplain{}{\bfseries\leftmark}]{\fancyplain{}{\bfseries\thepage}}
%\cfoot{}


% chapters of the document

\cleardoublepage
% Introduction
\phantomsection
\pagestyle{fancy}

\chapter{Introduction}
\onehalfspacing
%\addcontentsline{toc}{chapter}{Introduction}
%\section{Introduction}
\section{Justification of interest and relevance}

Currently, it has been established that on large scales, the structure of the universe is formed by a vast cosmic web primarily composed of dark matter \cite{Eniasto:2014}. The topology of this cosmic web consists of a network of filaments that enclosing large voids \cite{Anatole:2024}. These filaments, comprising mainly of dark matter halos, contain the baryonic matter composed of galaxy clusters and intergalactic matter. The largest and most populated clusters and superclusters of galaxies reside within these dark matter filaments, predominantly at their intersection points, and thus form the largest structures of the visible universe.

Determining the structure of these clusters is therefore crucial for understanding the matter distribution in the universe. This is where clustering algorithms come into play. For reasons that will be discussed further, the density-based algorithms may be the most appropriate for this study.

In this work we want to partially answer how density-based algorithms can be applied to determine how the matter is grouped in the universe.

\section{Personal motivation}

Given my background and strong interest in astrophysics and cosmology, upon entering the field of Data Science, it is easy to recognize the vast potential for applying the multiple Machine Learning (ML) techniques to these scientific domains. In particular, the study of the large-scale structure of the Universe is, without a doubt, one of the most fascinating topics in science today and where ML methods can find a large number of applications.

%% FUTURE INCLUSION OF M33 Image galaxy of our Local Group
%\begin{figure}[h]
%\centering
%\includegraphics[width=0.5\textwidth]{./figs/M33_2.jpg}
%\caption{M33: A galaxy belonging to our local group.}
%\label{fig:sample_figure}
%\end{figure}


\section{Goals definition}

There is a list of objectives I aim to achieve with this work:
\begin{itemize}
	\item Generate a visualization map of the data object in this study.
	\item Apply some density-based algorithm to galaxy and galaxy-groups datasets obtained from SDSS and 2DFGRS.
	\item Figure out which of these algorithms work better and determine the possible causes.
	\item Create a validation methods to obtain a hyperparameter tunning that optimize the group detection.
	\item Detect possible methods to improve this study.
\end{itemize}

It is also expected to achieve an approximation to actual groups of galaxies through the clusters obtained by the density-based methods.

% sustentability from now it will be leave out
% \subsection{Sustainability, diversity, and ethical/social challenges}

This section should assess the positive/negative impact of the project in the following dimensions. It is not required to reach a positive impact in any/all dimensions, but it is necessary to consider and discuss whether there is an impact or not from the beginning of the project.

\begin{description}
    \item[Sustainability] In the development of the project or during its entire lifecycle (e.g., deployment, retirement), does the output of this project have an impact on sustainability and/or ecological footprint (energy/resource consumption/savings, waste, pollution, depletion of raw materials)? Is it affected by laws or regulations on this matter? Considering another perspective, does it affect any of the Sustainable Development Goals (SDG) related to these dimensions? If it does not have any impact, either positive or negative, you should explain how you reached this conclusion and justify your answer.
    \item[Ethical behaviour and social responsibility] Is the outcome of the project too technical to have any positive/negative impact in ethical/social aspects? Does it have an impact on laws/regulations (data, privacy, labour, intellectual property, personal security, …)? Does it adhere to the deontological principles of the profession? Does it endanger/improve/worsen any job position? If it does not have any impact, either positive or negative, you should explain how you reached this conclusion and justify your answer.
    \item[Diversity, gender and human rights] Is the result of this project so technical that it has no positive/negative impact in terms of gender, diversity, or human rights? And in any laws/regulations? And in terms of accessibility, disability, ergonomics and/or information security? If it does not have any impact, either positive or negative, you should explain how you reached this conclusion and justify your answer.
\end{description}


\section{Methodology and project development}

Unsupervised algorithms, particularly density-based methods, will be applied to datasets drawn from surveys such as the SDSS and 2DFGRS to generate galaxy clustering models. These datasets are available at \cite{Blanton:2005}:

https://gax.sjtu.edu.cn/data/Group.html

To evaluate the performance of these models, the following criteria will be followed:

\begin{itemize}
    \item {Detected Clusters}: Groups successfully classified as clusters (often referred to as True Positives at the group level).

	\item {Undetected Clusters}: Groups not found or not identified in the clusters set (equivalent to True Negatives at the group level).

	\item {Cluster Purity Ratio}: The proportion of members in a detected cluster that actually belong to the underlying group/structure.

	\item {Cluster Completeness Ratio}: The proportion of members of a true underlying group/structure that are successfully included within the detected cluster.

	\item {Misclassified Members}: Individual data points (galaxies) belonging to a true group but classified outside of detected cluster. (Often referred to as False Negatives at the individual member level).

	\item {External Data Classified as Members}: Individual data points (galaxies) not belonging to a true group but erroneously classified inside a detected cluster. (Often referred to as False Positives at the individual member level).
\end{itemize}

For this study, R programming language will be used to deploy and run scripts within RStudio environment. Some python scripts can be used as well.

\section{Schedule}

A Gantt diagram in figure \ref{fig:stages_figure} shows the different stages of project development. The stages have been grouped on three sets:

\begin{figure}[h]
\centering
\includegraphics[width=1.0\textwidth]{./figs/gantt_conogram.png}
\caption{Stages of the project.}
\label{fig:stages_figure}
\end{figure}

\begin{itemize}
	\item The Planning stage (shown in green) involves gathering resources and defining the project's objectives.
	\item The technical development stage (shown in red) includes design, data processing, method application and outcomes assessment.
	\item Research and writing stages (shown in blue).
\end{itemize}	
Most of the time there is an overlap of stages, due following:
\begin{itemize}
	\item Initial stages: starting the project composed of several tasks.
	\item There are two stages of objectives: one defined at the start of the project and the other during implementation, which may lead to further development.
	\item Evaluating the outcomes as part of development.
\end{itemize}

\newpage

% Other subsections from template are still possible to be included
%\subsection{Summary of the outputs of the project}

It is not necessary to describe each output in detail: this will be done in the remaining chapters of the project.

\subsection{Brief description of the remaining chapters of the report}

Brief description of the contents of each chapter and their relationship with the rest of the project.

\section{Methods and resources}

In these sections, it is necessary to describe:

\begin{itemize}
    \item The most relevant aspects of the design and development of the project.
    \item The methodology used in the development process, describing the potential alternatives, the decisions that have been taken and the criteria used to take these decisions.
    \item A description of the products that have been created.
\end{itemize}

The structure of these sections may change according to the type of project being developed.

%\section{Results}

%-

%\section{Conclusions and future work}

%-

%\section{Glossary}

%-

\phantomsection
\pagestyle{fancy}
% \chapter{Bibliography}

\addcontentsline{toc}{chapter}{Bibliography}
\bibliographystyle{plain}
\bibliography{refs}

\section{Appendices}

\end{document}