\documentclass[12pt,a4paper,twoside]{book}  % add one blank page between sections
%\documentclass[12pt,a4paper,oneside]{book} % no more spaces between sections
\usepackage{graphicx}
\usepackage{setspace} %double spacing for text, single for captions, footnotes, etc.
%\usepackage{hypernat} %substitute for cite that allows hyperlinks
\usepackage{natbib} % substitute for 'hypernat' that works on Windows.
\usepackage[english]{babel}
\usepackage[utf8]{inputenc}
\usepackage{color}
\usepackage{hhline} % extended styles for tables
\usepackage{multirow}
\usepackage{subfigure}
\usepackage{acronym}
\usepackage{hyperref}
\usepackage{amsmath,amsmath,amssymb}
\usepackage{fancyhdr}
\usepackage{epsfig, amsmath}
\usepackage{algorithmic}
\usepackage{amsmath}
\usepackage[linesnumbered, ruled]{algorithm2e}
\usepackage{epigraph} 

% general settings
\hypersetup{
linktocpage=true,
colorlinks=true,
linkcolor=blue,
citecolor=blue,
}
\definecolor{Hgray}{gray}{0.6}

\newenvironment{definition}[1][Definition]{\begin{trivlist}
\item[\hskip \labelsep {\bfseries #1}]}{\end{trivlist}}

\setlength{\topmargin}{0cm}
\setlength{\textheight}{23cm}
\setlength{\textwidth}{17cm}
\setlength{\oddsidemargin}{0cm}
\setlength{\evensidemargin}{0cm}
\setlength{\headheight}{1cm}

% indicates that 'sub-sub-sections' are numbered and appear in the index
\setcounter{secnumdepth}{3}
\setcounter{tocdepth}{2}

% settings for code
\renewcommand{\algorithmicrequire}{\textbf{Input:}}
\renewcommand{\algorithmicensure}{\textbf{Output:}}

%%%%%%%%%%%%
% DOCUMENT %
%%%%%%%%%%%%
\begin{document}

\setcounter{section}{0} % Resets the section counter to 0 at the document's start
\renewcommand{\thesection}{\arabic{section}} % Changes the section numbering scheme

%%%%%%%%%%%%%%%%%%%%%%%%%%%%%%%%%%%%%
% Definning new variables			%
%%%%%%%%%%%%%%%%%%%%%%%%%%%%%%%%%%%%%
\newcommand{\titulo}{Applying Density-Based Algorithms to Galaxy Cluster Catalogs}
\newcommand{\subtitulo}{Unveiling Galaxy Structure with Unsupervised Clustering and 2PFC}
\newcommand{\autor}{Carlos Toro Peñas}
\newcommand{\supervisor}{Laura Ruiz Dern}
\newcommand{\area}{4: Data Science}
\newcommand{\profesor}{David Masip Rodo}
\newcommand{\keywords}{clustering, galaxy clusters, cosmology}
\newcommand{\keywordses}{clustering, cúmulos de galaxias, cosmología}

% cover page
\input{sections/0_title.tex}
\newpage

% abstract
\pagenumbering{roman}
\setcounter{page}{1}
\pagestyle{plain}
\singlespacing

%%%%%%%%%%%%%%%%
%%% CREDITS %%%
%%%%%%%%%%%%%%%%
\chapter*{Copyright}

\vspace{1cm}

\begin{figure}[ht]
\centering
\includegraphics[scale=1]{images/license.png}
\end{figure}

Copyright © 2025, Carlos Toro Peñas.
Attribution-NonCommercial-NoDerivs 3.0 Spain (CC BY-NC-ND 3.0 ES).

\href{https://creativecommons.org/licenses/by-nc-nd/3.0/es/}{3.0 Spain of CreativeCommons}.

%%%%%%%%%%%%%
%%% RECORD %%%
%%%%%%%%%%%%%
\chapter*{FINAL PROJECT RECORD}

\begin{table}[ht]
\centering{}
\renewcommand{\arraystretch}{2}
\begin{tabular}{r | l}
\hline
Title of the project: & \titulo \\
\hline
Author's name: & \autor \\
\hline
Collaborating teacher's name: & \supervisor \\
\hline
PRA's name: & \profesor\\
\hline
Delivery date (mm/yyyy): & MM/YYYY\\
\hline
Degree or program: & Master's degree in Data Sicience\\
\hline
Final Project area: & \area \\
\hline
Language of the project: & English\\
\hline
Keywords: & \keywords\\
\hline
\end{tabular}
\end{table}

%%%%%%%%%%%%%%%%%%%
%%% DEDICATION %%%
%%%%%%%%%%%%%%%%%%%
\chapter*{Dedication/Quote}


%%%%%%%%%%%%%%%%%%%
%%% Acknowledgements %%%
%%%%%%%%%%%%%%%%%%%
\chapter*{Acknowledgements}

%If deemed appropriate, mention the people, companies or institutions that have contributed to the realization of this project.

%%%%%%%%%%%%%%%%
%%% ABSTRACT %%%
%%%%%%%%%%%%%%%%
\chapter*{Abstract}
\addcontentsline{toc}{chapter}{Abstract}

\onehalfspacing

This work primary focuses on apply density-based algorithms to datasets from major surveys, including the Two-degree Field Galaxy Redshift Survey (2dFGRS) and the Sloan Digital Sky Survey (SDSS). The application will be followed by hyperparameter tuning and a performance assessment to identify the algorithms' strengths and weaknesses in actual galactic group detection. In the future, these methods may be applied to new surveys and other celestial regions.

\vspace{1.5cm}

\textbf{Keywords}:\keywords.

%%%%%%%%%%%%%%%%
%%% RESUMEN %%%
%%%%%%%%%%%%%%%%
\chapter*{Resumen}
\addcontentsline{toc}{chapter}{Resumen}

\onehalfspacing

Este trabajo tiene como tema central la aplicación de diferentes algoritmos basados en densidad a juegos de datos obtenidos de estudios como Two-degree Field Galaxy Redshift Survey (2dFGRS) y Sloan Digital Sky Survey (SDSS). Como resultado de esa aplicación, se hará un ajuste de híper-parámetros así como una evaluación del desempeño de tales algoritmos para analizar sus fortalezas y debilidades en su habilidad para la detección de cúmulos galácticos catalogados. Futuramente, se podrán realizar aplicaciones de estos algoritmos a nuevos estudios y otras regiones del firmamento.

\vspace{1.5cm}

\textbf{Palabras clave}: \keywordses.
\newpage

\pagestyle{fancy}
\renewcommand{\chaptermark}[1]{ \markboth{#1}{}}
\renewcommand{\sectionmark}[1]{\markright{ \thesection.\ #1}}

\lhead[\fancyplain{}{\bfseries\thepage}]{\fancyplain{}{\bfseries\rightmark}}
\rhead[\fancyplain{}{\bfseries\leftmark}]{\fancyplain{}{\bfseries\thepage}}
\cfoot{}

% table of contents
%\cleardoublepage
%\clearpage
\phantomsection
\addcontentsline{toc}{chapter}{Table of Contents}
\tableofcontents
% list of figures
%\cleardoublepage
%\clearpage
\phantomsection
\addcontentsline{toc}{chapter}{List of Figures}
\listoffigures

% list of tables
%\cleardoublepage
%\phantomsection
%\addcontentsline{toc}{chapter}{List of Tables}
%\listoftables

\thispagestyle{empty}

\pagenumbering{arabic}

%\pagestyle{fancy}
%\renewcommand{\chaptermark}[1]{ \markboth{#1}{}}
%\renewcommand{\sectionmark}[1]{\markright{ \thesection.\ #1}}
%\lhead[\fancyplain{}{\bfseries\thepage}]{\fancyplain{}{\bfseries\rightmark}}
%\rhead[\fancyplain{}{\bfseries\leftmark}]{\fancyplain{}{\bfseries\thepage}}
%\cfoot{}


% chapters of the document

%\cleardoublepage
\clearpage
% Introduction
\phantomsection
\pagestyle{fancy}

\chapter{Introduction}
\onehalfspacing
%\addcontentsline{toc}{chapter}{Introduction}
%\section{Introduction}
\section{Justification of interest and relevance}

Currently, it has been established that on large scales, the structure of the universe is formed by a vast cosmic web primarily composed of dark matter \cite{Eniasto:2014}. The topology of this cosmic web consists of a network of filaments that enclosing large voids \cite{Anatole:2024}. These filaments, comprising mainly of dark matter halos, contain the baryonic matter composed of galaxy clusters and intergalactic matter. The largest and most populated clusters and superclusters of galaxies reside within these dark matter filaments, predominantly at their intersection points, and thus form the largest structures of the visible universe.

Determining the structure of these clusters is therefore crucial for understanding the matter distribution in the universe. This is where clustering algorithms come into play. For reasons that will be discussed further, the density-based algorithms may be the most appropriate for this study.

In this work we want to partially answer how density-based algorithms can be applied to determine how the matter is grouped in the universe.

\section{Personal motivation}

Given my background and strong interest in astrophysics and cosmology, upon entering the field of Data Science, it is easy to recognize the vast potential for applying the multiple Machine Learning (ML) techniques to these scientific domains. In particular, the study of the large-scale structure of the Universe is, without a doubt, one of the most fascinating topics in science today and where ML methods can find a large number of applications.

%% FUTURE INCLUSION OF M33 Image galaxy of our Local Group
%\begin{figure}[h]
%\centering
%\includegraphics[width=0.5\textwidth]{./figs/M33_2.jpg}
%\caption{M33: A galaxy belonging to our local group.}
%\label{fig:sample_figure}
%\end{figure}


\section{Goals definition}

There is a list of objectives I aim to achieve with this work:
\begin{itemize}
	\item Generate a visualization map of the data object in this study.
	\item Apply some density-based algorithm to galaxy and galaxy-groups datasets obtained from SDSS and 2DFGRS.
	\item Figure out which of these algorithms work better and determine the possible causes.
	\item Create a validation methods to obtain a hyperparameter tunning that optimize the group detection.
	\item Detect possible methods to improve this study.
\end{itemize}

It is also expected to achieve an approximation to actual groups of galaxies through the clusters obtained by the density-based methods.

% sustentability from now it will be leave out
% \subsection{Sustainability, diversity, and ethical/social challenges}

This section should assess the positive/negative impact of the project in the following dimensions. It is not required to reach a positive impact in any/all dimensions, but it is necessary to consider and discuss whether there is an impact or not from the beginning of the project.

\begin{description}
    \item[Sustainability] In the development of the project or during its entire lifecycle (e.g., deployment, retirement), does the output of this project have an impact on sustainability and/or ecological footprint (energy/resource consumption/savings, waste, pollution, depletion of raw materials)? Is it affected by laws or regulations on this matter? Considering another perspective, does it affect any of the Sustainable Development Goals (SDG) related to these dimensions? If it does not have any impact, either positive or negative, you should explain how you reached this conclusion and justify your answer.
    \item[Ethical behaviour and social responsibility] Is the outcome of the project too technical to have any positive/negative impact in ethical/social aspects? Does it have an impact on laws/regulations (data, privacy, labour, intellectual property, personal security, …)? Does it adhere to the deontological principles of the profession? Does it endanger/improve/worsen any job position? If it does not have any impact, either positive or negative, you should explain how you reached this conclusion and justify your answer.
    \item[Diversity, gender and human rights] Is the result of this project so technical that it has no positive/negative impact in terms of gender, diversity, or human rights? And in any laws/regulations? And in terms of accessibility, disability, ergonomics and/or information security? If it does not have any impact, either positive or negative, you should explain how you reached this conclusion and justify your answer.
\end{description}


\section{Methodology and project development}

Unsupervised algorithms, particularly density-based methods, will be applied to datasets drawn from surveys such as the SDSS and 2DFGRS to generate galaxy clustering models. These datasets are available at \cite{Blanton:2005}:

https://gax.sjtu.edu.cn/data/Group.html

To evaluate the performance of these models, the following criteria will be followed:

\begin{itemize}
    \item {Detected Clusters}: Groups successfully classified as clusters (often referred to as True Positives at the group level).

	\item {Undetected Clusters}: Groups not found or not identified in the clusters set (equivalent to True Negatives at the group level).

	\item {Cluster Purity Ratio}: The proportion of members in a detected cluster that actually belong to the underlying group/structure.

	\item {Cluster Completeness Ratio}: The proportion of members of a true underlying group/structure that are successfully included within the detected cluster.

	\item {Misclassified Members}: Individual data points (galaxies) belonging to a true group but classified outside of detected cluster. (Often referred to as False Negatives at the individual member level).

	\item {External Data Classified as Members}: Individual data points (galaxies) not belonging to a true group but erroneously classified inside a detected cluster. (Often referred to as False Positives at the individual member level).
\end{itemize}

For this study, R programming language will be used to deploy and run scripts within RStudio environment. Some python scripts can be used as well.

\section{Schedule}

A Gantt diagram in figure \ref{fig:stages_figure} shows the different stages of project development. The stages have been grouped on three sets:

\begin{figure}[h]
\centering
\includegraphics[width=1.0\textwidth]{./figs/gantt_conogram.png}
\caption{Stages of the project.}
\label{fig:stages_figure}
\end{figure}

\begin{itemize}
	\item The Planning stage (shown in green) involves gathering resources and defining the project's objectives.
	\item The technical development stage (shown in red) includes design, data processing, method application and outcomes assessment.
	\item Research and writing stages (shown in blue).
\end{itemize}	
Most of the time there is an overlap of stages, due following:
\begin{itemize}
	\item Initial stages: starting the project composed of several tasks.
	\item There are two stages of objectives: one defined at the start of the project and the other during implementation, which may lead to further development.
	\item Evaluating the outcomes as part of development.
\end{itemize}

\newpage

\clearpage

% Introduction
% state of the art chapter
\phantomsection
\pagestyle{fancy}

\chapter{State of the art}
\onehalfspacing
%\addcontentsline{toc}{chapter}{State_of_art}
%\section{Introduction}

This chapter aims to update the reader on the current state of the research area addressed by this work. For this, we will focus on two different parts, first inherent challenges of the data collected by the surveys, and second, a will brief introduction to Machine Learning, whose techniques we will apply in this work.

\section{Surveys}

Fortunatelly, now a days we can acccess data belonging to several surveys, among others:

	- From 2dfGRS: Contains 245591 objects of then 221414 galaxies reliable data of galaxies.
	- From SDSS, we downloaded the DR7-modelC petrosian magnitude with 639359.


\section{Machine Learning applyied to cosmology}

We will, before presentig the results, give a brief description of the algorithms employed in this work.

\subsection{Supervised Learning}

Supervised learning focus on find patterns and relations within labeled data. The aim of Supervised Methods is to obtain some knowledge learned from given labeled-data in order to make predictions on some of the classes for new data. In other words given a set of data \( Z = (X, Y) = (X_1,.., X_n, Y_1, ...Y_m) \), we want to find a function F that holds \(Y=F(X) \).

A subset is taken from the original dataset, the so called training data \( Z_i = (X_i, Y_i) \). And then the problem is reduced to find the minumum of a loss function, which measures the difference between \( Y_i \) and \( F(X_i) \) 

The input of any supervised algorithms are the so called independant variables, the output are some other variables called dependant variables. Supervised algorithms used the information contained within the trainning data to learn relations between input variables and target vartiables.

The reason why we will not stop more on Supervised methods is because in our model we are not interested on making predictions on some target, instead we want to find patterns in some data distribution, which can lead to guess hom matter are shaped within a dimensional space.

\subsection{Unsupervised methods}

Unsupervised learning focus on analysis and model of data without using any tags or output classes. Instead a set of variables
\( X^T = (X_1,.., X_n) \) corresponding to n-observations is given, and then the method try to find interestring features from the observations.

From the unsupervised methods set we have:
	Clustering and segmentation: work by in distance and simillarity patterns, they can be divided as
	- Hierarquical: work by create sucessive partition of data and hierarquical tree creation called dendogram.   
	- Partitional: an initial set of clusters must be set in advance, the set is improved on an iterartive proccess. Example k-means
	Lot of algorithms of this class work by making assmptions of the data distribution, for example, k-means works by given a predefined list of k-groups and have some kind of probability distribution (for example Gaussian) for each group.
	This result on globular or spherical-shaped clusters and will not work well on non-spherical shapes ofdata. On a spatial distribution, any arbitrarity in the shape of data is expected, for exmaple, linear, stellar-like or polygonal shapes and so on. That is why this algorithm is not suitable for our analysis.
	
	
	Density-based:
		The key feature is the no-assumption on the distribution of the data is made. Density-based algorithm works by create clusters in dense areas separated by sparser areas. That is why this algorithms are more suitable, provided that no assumtion is made on the shape of dense areas.
		Other feature of density-based methods is the ability to detect outliers or noise points. These points generate low-density areas which separes more dense areas which result in clusters.
	Among the density based we selected OPTICS, DBSCAN and HDBSCAN for this study, we will present in the next an overview of each.

\subsection{OPTICS}
	Ordering Points to Identify Cluster Structure,OPTICS is an unsupervised algorithm that works by ordering the points in function of the so-called reachability distance.
	In fact, OPTICS output is not a cluster but a graph call the reachability plot.
	Lets suposse we have a set call S, lets see some definitions:
	
	\begin{itemize}
		\item An eps-neighborhood of a point p in S is \(NE_{\epsilon}(p) =\{q \in S : dist(p, q) \leq eps \} \). Then any eps-neighborhood of p is said to be dense if 
		\( |NE_{\epsilon}(p)| \geq minPts \).
	
		\item The core\_distance of a given point p is the minimum \(\epsilon\) such us \(NE_{\epsilon}(p)\) is dense, in other words:
		
		$$ core_-distance(p) = min \{ \epsilon : |NE_{\epsilon}(p)| \geq minPts \} $$
	
		\item A point is said to be a core-point when \(NE_{\epsilon}'(p) \) is dense and  \( \epsilon' \leq \epsilon \), finally,
	
	
		\item The reachability distance from q regading a core-point g is the maximum of the two: core-distance and euclidean  distance, in other words:

		$$ reachability-distance(p, q) = max \{ core_-distance(p), dist(p, q) \} $$
		
	\end{itemize}
	
	\begin{figure}[h]
	\centering
	\includegraphics[width=0.5\textwidth]{./figs/core-distance.jpg}
	\caption{Core and reachability distance obtained from \cite{Rhys:2020}.}
	\label{fig:optics_figure}
	\end{figure}
	
	Note that reachability-distance is only defined with respect to a core-point. We can see an useful example of both reachability and core distances in the figure \ref{fig:optics_figure}.

	
	OPTICS work by seting up two parameters for each point: 
		- \( \epsilon \): 
		-  minPts
	For example \ref{fig:reach1} shows a random-generated set points arround four fixed points in [0,1]x[0,1]
	
	\begin{figure}[h]
	\centering
	\includegraphics[width=0.5\textwidth]{./figs/reach1.png}
	\caption{An example of data set in plane \( \mathbb{R}^2\).}
	\label{fig:reach1}
	\end{figure}
	
	The \ref{fig:reach2} shows the reachability plot corresponding to a this set.
	
	\begin{figure}[h]
	\centering
	\includegraphics[width=0.5\textwidth]{./figs/reach2.png}
	\caption{Example of OPTICS reachability plot}
	\label{fig:reach2}
	\end{figure}
	
\subsection{DBSCAN}
DBSCAN is an algorithm that uses some of OPTICS concepts to extract clusters, but is need to ad some more difinitions, given a set S MinPts and Eps, lets be p a core-point of S then:
	
\begin{itemize}
		\item a point q is said to be direct density reachable with respect to Eps and MinPts from p if \(q \in NE_{\epsilon}(p)\). Note that \(|NE_{\epsilon}(p)| \geq MinPts \).
		\item a point q is said to be density-reachable with respect to Eps and MinPts if there exists a chanin of points \(p_1, ..., p_n,\_\_ p=p_1, p_n=q \) such us  \( p_{i+1} \) is direct density reachable from \( p_i \).
		\item A point p is density connected to a point q with respect to Eps and MinPts if there is thirdth point o such that both p and q are density-reachable from o with respect to Eps and MinPts.
\end{itemize}		
Then a cluster C is a subset of S satisfying:
	\begin{itemize}
		\item \(\forall p, q, in C\)  p is density-connected from q with respect to Eps and MinPts.
		\item \(\forall p, q, in C\) if q is density reachable from p with respect to Eps and MinPts then \(q \in C \).
	\end{itemize}

Then with this method, DBSCAN creates a set of clusters \(C_1, ..., C_k\). All points in S are classified as
		1. Core-points: points with a dense neighborhood.
		2. Border-points: points belonging to a cluster but without a dense neighborhood.
		3. Noise points: points do not belonging to any cluster.
		
To find a cluster, DBSCAN starts with an arbitrary database point p and retrieves all points density-reachable from p with respect to Eps and MinPts. If p is not a core point, no points are density-reachable from p and DBSCAN assigns p to the noise and applies the same procedure to the next database point. If p is actually a border point of some cluster C, it will later be reached when collecting all the points density-reachable from some core point of C and will then be (re-)assigned to C. The algorithm terminates when all points have been assigned to a cluster or to the noise.

\begin{algorithm}[H]
\SetKwInOut{Input}{Input}
\SetKwInOut{Output}{Output}
\SetKwFunction{RegionQuery}{RegionQuery}
\SetKwFunction{ExpandCluster}{ExpandCluster}

\Input{Dataset $D$, $\epsilon$ (epsilon), $\text{MinPts}$ (minimum points)}
\Output{Set of clusters $C$, with noise points unassigned}

\BlankLine

$C \leftarrow 0$ \tcp{Cluster counter}
\For{each point $P$ in $D$}{
    \If{$P$ is unvisited}{
        mark $P$ as visited\;
        $N \leftarrow \RegionQuery(D, P, \epsilon)$\;
        \If{$|N| < \text{MinPts}$}{
            mark $P$ as \textbf{Noise}\;
        }
        \Else{
            $C \leftarrow C + 1$\;
            $\ExpandCluster(D, P, N, C, \epsilon, \text{MinPts})$\;
        }
    }
}

\BlankLine

\SetKwProg{Fn}{Function}{}{end}
\Fn{\ExpandCluster{$D, P, N, C, \epsilon, \text{MinPts}$}}{
    assign $P$ to cluster $C$\;
    \For{each point $P'$ in $N$}{
        \If{$P'$ is unvisited}{
            mark $P'$ as visited\;
            $N' \leftarrow \RegionQuery(D, P', \epsilon)$\;
            \If{$|N'| \geq \text{MinPts}$}{
                $N \leftarrow N \cup N'$\;
            }
        }
        \If{$P'$ is not yet assigned to a cluster}{
            assign $P'$ to cluster $C$\;
        }
    }
}

\BlankLine

\Fn{\RegionQuery{$D, P, \epsilon$}}{
    \KwRet{all points $P' \in D$ such that $\text{distance}(P, P') \leq \epsilon$}
}

\caption{The DBSCAN Algorithm}
\label{alg:dbscan}
\end{algorithm}


\subsection{HDBSCAN}

\subsection{Non density-based algorithims}



\newpage
% Other subsections from template are still possible to be included
%\input{/sections/3_other_subsections.tex}
\phantomsection
\pagestyle{fancy}

\chapter{Implementation}
\onehalfspacing
%\addcontentsline{toc}{chapter}{State_of_art}
%\section{Introduction}


\section{ETL processing of datasets}

This section describes the Data Engineering Pipeline that converts the astronomical raw data into the machine learning-ready format you used for your 2dFGRS analysis.

Our Python framework acts as a bridge between the raw observational catalog and the unsupervised learning models. By producing a sanitized CSV with pre-calculated, scaled Cartesian coordinates to operate with maximum efficiency and physical accuracy which data definition is shown in table \ref{table:data}.

The final objective of this pipeline is to associate individual galaxies with their respective Dark Matter Halos or larger structures. To achieve this, we follow three basic steps to merge the algorithmic output with the physical catalog:

\begin{enumerate}
	\item Format the galaxy catalog to a CSV file.
	\item Format the group catalog to a CSV file.
	\item Merge galaxy and group catalog and transform coordinates and distances.
\end{enumerate}	

The synthesis results in a unified dataset formatted for computational efficiency. The header of this processed file, which contains the spatial and environmental metadata for each galaxy, is displayed in Figure \ref{fig:cat}.

Due to the varying structures of the survey catalogs, the data acquisition and preparation phase is divided into distinct modules to ensure inter-survey compatibility.

\subsection{2dF Galaxy Redshift Survey (2dfGRS)} \label{data:2d}

\begin{enumerate}
	\item \textit{2dfGRS.dat}: Which comprises 245,591 individual galaxy entries. To ensure high-fidelity measurements and minimize redshift uncertainty, we applied a quality constraint of $Q \geq 3$, excluding objects with poorly determined spectral features or low signal-to-noise ratios
	
	\item \textit{group\_members}: a supplementary group-membership file consisting of 104,913 galaxies.
\end{enumerate}	


\subsection{Sloan Digital Sky Survey (SDSS)}

Among the diverse datasets provided by the SDSS archive, the imodelC\_1 file was identified as the most suitable for this analysis. It contains the required astrometric and photometric parameters to accurately cross-match with our random catalogs, allowing for a robust calculation of the large-scale structure signal.

\begin{enumerate}
	\item \textit{SDSS7}: Galaxy catalog of the survey.
	\item \textit{imodelC\_1} Comprises 245,591 entries for each galaxy. (again we applied a quality constraint of $Q \geq 3$.)
\end{enumerate}	
	
Our pipeline performs a multi-source integration of the raw data files, executing the necessary joins and quality filters to produce a unified CSV file optimized for clustering analysis. A representative sample of this finalized data product is provided in Figure \ref{fig:cat}, demonstrating the successful synthesis of spatial and environmental metadata.

\begin{table}[]
\centering
\begin{tabular}{|l|l|l|}
\hline
\textbf{Field-Name} & \textbf{FDescription} & \textbf{Field-Type} \\
\hline
 GAL\_ID & ID of galaxy en each catalog & numerical \\ \hline
 ra & Right ascension coordinate  & decimal \\  \hline
 dec & Declination coordinate  & decimal    \\ \hline
 x & X cartesian coordinate & decimal    \\ \hline
 y & Y cartesian coordinate & decimal    \\ \hline
 z & Z cartesian coordinate & decimal    \\ \hline
 redshift & cell8 & decimal \\ \hline
 dist & Raw distance value & decimal  \\ \hline
 GROUP\_ID & id-group galaxy belongs to & numerical \\ \hline
\end{tabular}
\caption{Datasheet metadata.}
\label{table:data}
\end{table}

\subsection{Real Space Galaxy Catalogue}
Finally, we incorporated the SDSS 'Real Space Galaxy Catalogue', which provides galaxy coordinates reconstructed to account for redshift distortions as we explained in sections \ref{section:slos} and \ref{section:real} . This enables a robust comparison between observed clustering and theoretical predictions by isolating the isotropic real-space correlation function.

	\begin{figure}[h]
	\centering
	\includegraphics[width=1\textwidth]{./figs/catalog.jpg}
	\caption{ Final format of dataset }
	\label{fig:cat}
	\end{figure}
	

\section{The two-point correlation function (2pcf)}

Drawing on the comparative analysis provided by \cite{Kerscher:2000}, we implement four distinct estimators to quantify the clustering signal. This selection allows for a rigorous cross-examination of the statistical bias and variance inherent in each method when applied to large-scale galaxy catalogs.

Natural
\begin{equation} \label{eq:43}
	\widehat{\xi} _{N}= \frac{DD}{RR} - 1
\end{equation}.

Davids and Peebels:
\begin{equation} \label{eq:41}
	\widehat{\xi} _{DP}= \frac{DD}{DR} - 1
\end{equation}.

Hamilton:
\begin{equation} \label{eq:42}
	\widehat{\xi} _{Ha}= \frac{DD \, RR}{DR^2}
\end{equation}.

Landy and Szalay:
\begin{equation} \label{eq:44}
	\widehat{\xi} _{LS}= \frac{DD -2DR +RR}{RR}
\end{equation}.

Upon applying these estimators to the 2dFGRS dataset, as we will se in  \ref{2pcf} we successfully recovered a distinct clustering feature at approximately $100 \ h^{-1}\text{Mpc}$ (see figures \ref{estimators1} and \ref{estimators2}. This signal is statistically consistent with the predicted scale of Baryon Acoustic Oscillations (BAOs), representing a detection of the primordial acoustic horizon in the late-time galaxy distribution.


	\begin{figure}[h]
	\centering
	\includegraphics[width=1\textwidth]{./figs/estimators1.jpg}
	\caption{ Natural and Hamilton estimators measured on the 2dFGRS sample. }
	\label{fig:estimators1}
	\end{figure}
	
	\begin{figure}[h]
	\centering
	\includegraphics[width=1\textwidth]{./figs/estimators2.jpg}
	\caption{ David and Peebels and Landy and Szalay estimators for 2dFGRS sample. }
	\label{fig:estimators2}
	\end{figure}
	
\section{Application of density-based algorithms to datasets}

We performed a comparative evaluation of several density-based clustering frameworks to assess their capability in recovering the physical halo distribution. The algorithms were selected based on their distinct approaches to density reachability, hierarchical extraction, and noise handling

We tested several algorithms in order to obtain a model of density clustering both for non-scaled and scaled data to ensure that the density metrics are isotropic and not biased by the differing scales of the coordinate axes.
to prevent .

\begin{itemize}
    \item OPTICS: Utilized to generate a reachability plot, providing a visualization of the hierarchical density structure and identifying the spatial ordering of galaxies.
	
	\item OPTICSXi: An extension of OPTICS used to extract clusters in a hierarchical mode by identifying steep density gradients (the $\xi$ parameter), allowing for the detection of clusters with varying densities.
	
	\item DBSCAN: Implemented as a baseline density-based method to identify clusters as density-connected components based on a fixed global proximity threshold ($\epsilon$).
	
	\item HDBSCAN: A robust hierarchical implementation that constructs a spanning tree to find stable clusters across all possible density scales, making it highly effective for multi-scale cosmological distributions.
	
	\item DPC (Density Peaks Clustering): Employed to identify clusters based on the detection of local density maxima and their relative distance from other high-density peaks, which is physically analogous to identifying halo centers.
		
	\item sOPTICS and sDBSCAN: These variants account for line-of-sight positional uncertainties due to redshift space distortions with the modified distances as explained at \ref{section:slos}
\end{itemize} \label{sel:algorithms}


It is important to emphasize that this study departs from traditional unsupervised clustering objectives, such as minimizing intra-cluster variance via the Elbow Method. Instead, we treat the Halo-based group distribution as a physical ground truth. Consequently, many standard clustering algorithms —and their default hyperparameter configurations— may fail to yield results consistent with our model of virialized galaxy groups, as they are not intrinsically designed to account for the specific density profiles of dark matter halos.

The performance of each algorithm is evaluated based on its Recovery Rate of known halo members. We prioritize models that maximize the completeness and purity of identified groups relative to the 2dFGRS/SDSS group catalogs, rather than those that simply minimize global silhouette scores.

Guided by the validation protocols established in \cite{Hai-Xia-Ma:2025} we employ the following metrics to assess the topological and member-wise similarity between the density-based models ($C$) and the halo-based ground truth ($H$):

We define:
\begin{itemize}
	\item $total\_in\_cluster$: number of elements in output-cluster.
	\item $total\_in\_cluster\_group$: number of elements from orignal group present in an output-cluster. 
	\item $total\_in\_cluster_group$: number of elements from orignal group present in an output-cluster.
\end{itemize}
	
\begin{equation} \label{eq:purity}
	 Purity = \mathcal{P} = \frac{total\_in\_cluster\_group}{total\_in\_cluster}
% $$\mathcal{P} = \frac{|C \cap H|}{|C|}$$
\end{equation}.

\begin{equation} \label{eq:completeness}
	 Completeness = \mathcal{C} = \frac{total\_in\_cluster\_group}{total\_in\_group} 
%$$\mathcal{C} = \frac{|C \cap H|}{|H|}$$
\end{equation}.

Purity is also called \textbf{Precision} and completeness \textbf{Sensitivity} and also \textbf{Recall}.

Following the categorical framework of \cite{Hai-Xia-Ma:2025} we define the thresholds:

\begin{itemize}
	\item Purity Threshold ($\mathcal{P} \geq 2/3$): An output-cluster is defined as Pure if at least 66.7% of its constituent galaxies originate from the same parent dark matter halo. 
	
	\item Completeness Threshold ($\mathcal{C} \geq 1/2$): An output-cluster is defined as Complete if it successfully captures at least 50% of the galaxies belonging to the true physical halo (or original group). This ensures that the algorithm has recovered the core structure of the virialized group.
\end{itemize}

With this concepts we evaluate the following ratios:
  
\begin{equation} \label{eq:purity_rate}
	F_{p} = \frac{N_{pure}}{N_{clusters}}
\end{equation}.

\begin{equation} \label{eq:completeness_rate}
	F_{c} = \frac{N_{complete}}{N_{clusters}}
\end{equation}.

Fr only for complete and pure output-clusters:
% $$R_M = \frac{N_{pure \cap complete}}{N_{total\_clusters}}$$
\begin{equation} \label{eq:completeness_pure_rate}
	F_{r} = \frac{N_{complete} + N_{pure}}{N_{original\_groups}}
\end{equation}.


%\begin{equation} \label{eq:pure_complete_rate}
%	Fr = \frac{total\_in\_group}{number\_non\_isolated\_galaxies}
%\end{equation}.


\subsection{Application to 2dFGRS catalog} \label{results:2dfgrs}

By applying the success-matching protocol derived from \cite{Hai-Xia-Ma:2025}, we evaluated the performance of each density-based configuration. The table \ref{table:2df} summarizes the ability of each algorithm to recover the underlying group or  halo distribution within the 2dFGRS survey volume.

\begin{table}[]
% \resizebox{\textwidth}{!}{
\centering
\scalebox{0.7}{
% \setlength{\tabcolsep}{100pt} % Default value: 6pt
\renewcommand{\arraystretch}{1.5} % Default value: 1
\begin{tabular}{c c c c c}
\hline
\textbf{Algorithm} & \textbf{Hyperparameter} & \textbf{Data Sample} & \textbf{Outcomes Sample} & \textbf{Conclusion}  \\
\hline
 DBSCAN & $\begin{matrix} \epsilon = 6 \times  10^{-4} \\ minPts=5 \end{matrix} $ & Non-scaled & $\begin{matrix}P=0.65 \\ C=0.87 \\ R= 0.42 \\ U=21 \end{matrix}$ & Good in cluster detection  \\ \hline
 
 HDBSCAN & - & - & - & Not good in cluster detection. \\  \hline
 
 DPC & $\begin{matrix} \rho = 8.4 \times  10^{-4} \\ \delta = 0.9985 \end{matrix} $  & Non-scaled & - & Good in cluster center detection \\  \hline
 
 sOPTICS &  $\begin{matrix} \epsilon = 11 \times  10^{-5} \\ minPts=5 \end{matrix} $  & Non-scaled & $\begin{matrix} P=0.84 \\C= 0.84 \\ R=0.86 \\ U=12 \end{matrix}$ \\ & Best results \\ \hline
  
  DBSCAN & $\begin{matrix} \epsilon = 6 \times  10^{-4} \\ minPts=5 \end{matrix} $ & Scaled & $\begin{matrix}P=0.72 \\ C=0.81 \\ R= 0.41 \\ U=22 \end{matrix}$ & Good in cluster detection  \\ \hline
 
 OPTICS & - & - & - & Good in cluster detection. \\  \hline
\end{tabular}}
\caption{Results on 2dFGRS sample.}
\label{table:2df}
\end{table}

\subsection{Application to SDSS catalog} \label{results:sdss}

We applied the dentisy-based algorithms to the SDSS catalog shown in section \ref{data:sdss}.

The outcomes are showed in table \cite{table:sdss}. Same results of application in \label{results:2dfgrs} can be shown as well, so sOPTICS is the best fitting algorithm which meets the results showed in \cite{Hai-Xia-Ma:2025}. 

\begin{table}[]
% \resizebox{\textwidth}{!}{
\centering
\scalebox{0.7}{
% \setlength{\tabcolsep}{100pt} % Default value: 6pt
\renewcommand{\arraystretch}{1.5} % Default value: 1
\begin{tabular}{c c c c c}
\hline
\textbf{Algorithm} & \textbf{Hyperparameter} & \textbf{Data Sample} & \textbf{Outcomes Sample} & \textbf{Conclusion}  \\
\hline
 DBSCAN & $\begin{matrix} \epsilon = 6 \times  10^{-4} \\ minPts=5 \end{matrix} $ & Non-scaled & $\begin{matrix}P=0.65 \\ C=0.87 \\ R= 0.42 \\ U=21 \end{matrix}$ & Good in cluster detection  \\ \hline
 
 HDBSCAN & - & - & - & Not good in cluster detection. \\  \hline
 
 DPC & $\begin{matrix} \rho = 8.4 \times  10^{-4} \\ \delta = 0.9985 \end{matrix} $  & Non-scaled & - & Good in cluster center detection \\  \hline
 
 sOPTICS &  $\begin{matrix} \epsilon = 11.0 \times 10^{-5} \\ minPts=5 \end{matrix} $  & Non-scaled & $\begin{matrix} P=0.84 \\C= 0.84 \\ R=0.86 \\ U=12 \end{matrix}$ \\ & Best results \\ \hline
  
  DBSCAN & $\begin{matrix} \epsilon = 6 \times  10^{-4} \\ minPts=5 \end{matrix} $ & Scaled & $\begin{matrix}P=0.72 \\ C=0.81 \\ R= 0.41 \\ U=22 \end{matrix}$ & Good in cluster detection  \\ \hline
 
 OPTICS & - & - & - & Good in cluster detection. \\  \hline
\end{tabular}}
\caption{Results on SDSS sample.}
\label{table:2df}
\end{table}

\subsection{Application to SDSS Real Space Galaxy Catalogue}

We also applied same density-based algorithms to the SDSS Real Space Galaxy Catalogue as shown article \cite{Shi:2016}.

Once corrected the distorsions we talked in \ref{section:real} all algorithms resulting in a better fitting model detection  compared with the ones shown in the \ref{results:sdss} section which contain space distorsions as showed at \ref{section:real}.

\begin{table}[]
% \resizebox{\textwidth}{!}{
\centering
\scalebox{0.7}{
% \setlength{\tabcolsep}{100pt} % Default value: 6pt
\renewcommand{\arraystretch}{1.5} % Default value: 1
\begin{tabular}{c c c c c}
\hline
\textbf{Algorithm} & \textbf{Hyperparameter} & \textbf{Data Sample} & \textbf{Outcomes Sample} & \textbf{Conclusion}  \\
\hline
 DBSCAN & $\begin{matrix} \epsilon = 3 \times  10^{-4} \\ minPts=5 \end{matrix} $ & Non-scaled & $\begin{matrix}P=0.83 \\ C=0.92 \\ R= 0.99 \\ U=6 \end{matrix}$ & Good in cluster detection  \\ \hline
 
 HDBSCAN & - & - & - & Not good in cluster detection. \\  \hline
 
 DPC & $\begin{matrix} \rho = 8.5 \times 10^{-4} \\ \delta = 0.9986 \end{matrix} $  & Non-scaled & - & 100\% in cluster center detection \\  \hline
   
  DBSCAN & $\begin{matrix} \epsilon = 2.6 \times 10^{-2} \\ minPts=5 \end{matrix} $ & Scaled & $\begin{matrix}P=0.88 \\ C=0.88 \\ R= 0.97 \\ U=7 \end{matrix}$ & Good in cluster detection  \\ \hline
 
 OPTICS & - & Scaled & - & Good in cluster detection. \\  \hline
\end{tabular}}
\caption{Results on SDSS Real Space Galaxy Catalogue sample \cite{Shi:2016}.}
\label{table:real}
\end{table}

\subsection{Impact of Standardization (results on scaled data)}

While the spatial coordinates were initially defined on a consistent physical, to ensure that our density metrics remained isotropic and independent of the varying scales of the coordinate axes, we implemented a standardization protocol (Z-score normalization).

By transforming the spatial features to have a mean of zero and unit variance—calculated as $z = (x - \mu) / \sigma$—we eliminated the numerical bias inherent in raw coordinate ranges. 

This preprocessing step yielded a measurable increase in cluster detection sensitivity across the 2dFGRS, SDSS, and Real-Space Galaxy catalogues as we can see in tables \label{table:2df}, \label{table:sdss} and \label{table:real}. This confirms that enforcing numerical isotropy is essential for correctly identifying groups in the all geometries.

\section{Conclusions of density-based algorithm appllication on different catalogs}\label{conclusions}

The results of our comparative analysis along different catalogs can be summarized in the following key findings:

\begin{enumerate}
	\item Superiority of Standardized Data: The application of Z-score normalization was the single most influential factor in improving model performance. By ensuring numerical isotropy, the scaled variants (sHDBSCAN, sDBSCAN, and sOPTICS) consistently outperformed their raw-coordinate counterparts across all metrics ($P$, $C$, $R$, and $U$).
	
	\item Best Performance of sOPTICS (sDBSCAN): Among the tested frameworks, sDBSCAN emerged as the most robust model for recovering the underlying halo distribution. It achieved the highest Valid Match Ratios.
	
	\item Modifying the distance in an elongated way along the Line of Sight improves the detection as sOPTICS algorithm which fit with \cite{Hai-Xia-Ma:2025}.
	
	\item Elbow method: We observed that traditional geometric optimizations, such as the Elbow Method, are unsuitable for clustering detection. As it was said, our results confirm that local density reachability is a more physically accurate proxy for gravitational binding than global variance minimization.
	
\end{enumerate}

\section{Applying 2PCF Estimators to the 2dFGRS Dataset}\label{2pcf}

This is another mode to analysis the density of galaxies across the Universe: we will move from a discrete classification (deciding which galaxy belongs to which group) to statistical distribution (measuring how galaxies "crowd" together across the entire manifold). While clustering algorithms tell where the groups are, the Two-Point Correlation Function (2PCF) deals with the probability of finding galaxies at specific distances from each other.

We created a python notebook based on the dataset shown in section \ref{data:2d}. Loaded the 2dFGRS dataset we take a sample to enable us to compare the distribution in the space of galaxies contained in the sample. By contrasting the 2dFGRS sample with a synthetic Poisson we expect this approach may convey information about the geometry of the matter across the Universe.

One interestring point is the use of scipy.spatial.KDTree which allow to improve the time response and calculations by providing an index in a set of k-dimensional space. This indexing strategy was pivotal for both the density-based cluster extraction and the 2PCF pair-counting, reducing the computational complexity from quadratic to logarithmic scales. This ensured that our hyperparameter grid search remained performant even when processing 3D comoving coordinates across 100 Mpc/h scales.

The obtained results are interestring because we can observe a peak  over the $100h^{-1} Mp$ which can be explained by Baryonic Acoustic Oscillations, this result provides definitive evidence that our density-based clustering methodology recovers the fundamental 'fingerprint' of the early Universe still visible in the galaxy distribution and consistent with  predictions of the $\lambda-CDM$ model.

The BAOs is also one more demonstration of the dark matter presence, sparse in the Universe, shaping it, creating the halos where clusters shown in this study reside.

Finally, our results show that the clusters identified by sDBSCAN and sOPTICS are not random associations, but are the physical manifestations of galaxies residing within the gravitational potential wells of Dark Matter halos, whose distribution was dictated by the sound horizon of the early Universe."

\phantomsection
\pagestyle{fancy}

\chapter{Results, conclusions and future works}
\onehalfspacing
%\addcontentsline{toc}{chapter}{State_of_art}
%\section{Introduction}

In this section provides a brief synthesis of the research, demonstrating how the combination of R-based density clustering and Python-based statistical estimators effectively maps the large-scale structure of the Universe.

\section{Results of application density-based algorithms}
\subsection{2dFGRS sample} \label{results:2dfgrs}

By applying the success-matching protocol derived from Section \label{sect:aplication} and  \cite{Hai-Xia-Ma:2025}, we evaluated the performance of each density-based configuration. The table \ref{table:2df} summarizes the ability of each algorithm to recover the underlying group or  halo distribution within the 2dFGRS survey volume. Notably, sOPTICS achieved the highest Recovery ($\mathcal{R}$) rates. While other algorithms as DBSCAN demonstrated acceptable in Purity or Completeness rates, they proved less effective overall due to significantly lower Recovery rates, failing to identify a representative fraction of the total group population.


\begin{table}[]
% \resizebox{\textwidth}{!}{
\centering
\scalebox{0.7}{
% \setlength{\tabcolsep}{100pt} % Default value: 6pt
\renewcommand{\arraystretch}{1.5} % Default value: 1
\begin{tabular}{c c c c c}
\hline
\textbf{Algorithm} & \textbf{Hyperparameter} & \textbf{Data Sample} & \textbf{Outcomes Sample} & \textbf{Conclusion}  \\
\hline
 DBSCAN & $\begin{matrix} \epsilon = 6 \times  10^{-4} \\ minPts=5 \end{matrix} $ & Non-scaled & $\begin{matrix}P=0.65 \\ C=0.87 \\ R= 0.42 \\ U=21 \end{matrix}$ &  $\begin{matrix} \text{Reasonable cluster detection.} \\ \text{ Low recovery\-rate.} \end{matrix}$  \\ \hline
 
 HDBSCAN & - & - & - & Not good in cluster detection. \\  \hline
 
 DPC & $\begin{matrix} \rho = 8.3 \times  10^{-4} \\ \delta = 0.9985 \end{matrix} $  & Non-scaled & - & Good in cluster center detection \\  \hline
 
 sOPTICS &  $\begin{matrix} \epsilon = 11 \times  10^{-5} \\ minPts=5 \end{matrix} $  & Non-scaled & $\begin{matrix} P=0.84 \\C= 0.84 \\ R=0.86 \\ U=12 \end{matrix}$ \\ & Best results \\ \hline
  
  DBSCAN & $\begin{matrix} \epsilon = 6 \times  10^{-4} \\ minPts=5 \end{matrix} $ & Scaled & $\begin{matrix}P=0.72 \\ C=0.81 \\ R= 0.41 \\ U=22 \end{matrix}$ &  $\begin{matrix} \text{Acceptable cluster detection.} \\ \text{ Low recovery\-rate.} \end{matrix}$  \\ \hline
 
 OPTICS & - & - & - & Good in reachability plot. \\  \hline
\end{tabular}}
\caption{Results on 2dFGRS sample.}
\label{table:2df}
\end{table}

\subsection{SDSS sample} \label{results:sdss}

Density-based algorithms to the SDSS catalog shown in section \ref{data:sdss}.

The results of the analysis are summarized in Table \ref{table:sdss}. Comparable performance trends are observed in the application to the 2dFGRS catalog (Section \ref{results:2dfgrs}). Across both datasets, sOPTICS emerges as the most effective algorithm, demonstrating the highest alignment with the benchmark results reported by \cite{Hai-Xia-Ma:2025}. Its success is attributed to its ability to resolve the complex density profiles of galaxy groups within the reconstructed cosmic web.

\begin{table}[]
% \resizebox{\textwidth}{!}{
\centering
\scalebox{0.7}{
% \setlength{\tabcolsep}{100pt} % Default value: 6pt
\renewcommand{\arraystretch}{1.5} % Default value: 1
\begin{tabular}{c c c c c}
\hline
\textbf{Algorithm} & \textbf{Hyperparameter} & \textbf{Data Sample} & \textbf{Outcomes Sample} & \textbf{Conclusion}  \\
\hline
 DBSCAN & $\begin{matrix} \epsilon = 3.7 \times  10^{-4} \\ minPts=5 \end{matrix} $ & Non-scaled & $\begin{matrix}P=0.68 \\ C=0.68 \\ R= 0.25 \\ U=30 \end{matrix}$ & $\begin{matrix} \text{Reasonable cluster detection.} \\ \text{ Low recovery\-rate.} \end{matrix}$ \\ \hline
 
 HDBSCAN & - & - & - & Not suitable in cluster detection. \\  \hline
 
 DPC & $\begin{matrix} \rho = 8.0 \times  10^{-4} \\ \delta = 0.9985 \end{matrix} $  & Non-scaled & - & Good in cluster center detection \\  \hline
 
 sOPTICS &  $\begin{matrix} \epsilon = 11.0 \times 10^{-5} \\ minPts=5 \end{matrix} $  & Non-scaled & $\begin{matrix} P=0.82 \\C= 0.72 \\ R=0.69 \\ U=12 \end{matrix}$ & Best results \\ \hline
  
 DBSCAN & $\begin{matrix} \epsilon = 6 \times  10^{-4} \\ minPts=5 \end{matrix} $ & Scaled & $\begin{matrix}P=0.69 \\ C=0.69 \\ R= 0.57 \\ U=16 \end{matrix}$ &  $\begin{matrix} \text{Good in cluster detection.} \\ \text{ Low recovery\-rate.} \end{matrix}$  \\ \hline
 
 OPTICS & - & - & - & Good in reachability plot. \\  \hline
\end{tabular}}
\caption{Results on SDSS sample.}
\label{table:sdss}
\end{table}

\subsection{SDSS Real Space Galaxy sample}

Finally, the applied same density-based algorithms were applied to the SDSS Real Space Galaxy Catalogue as shown article \cite{Shi:2016} which results are shwon in the table \ref{table:real}.

Following the correction of the distortions detailed in Section \ref{section:real}, all algorithms exhibited significantly improved performance in halo detection. This represents a marked enhancement over the results presented in Section \ref{results:sdss}, where the presence of redshift space distortions hindered the algorithms' ability to accurately reconstruct physical structures.

It is important to note that the application of sLOS (scaled Line-of-Sight) algorithms is unnecessary within this framework. Since the Re-Real space reconstruction already accounts for and mitigates line-of-sight distortions.

\begin{table}[]
% \resizebox{\textwidth}{!}{
\centering
\scalebox{0.7}{
% \setlength{\tabcolsep}{100pt} % Default value: 6pt
\renewcommand{\arraystretch}{1.5} % Default value: 1
\begin{tabular}{c c c c c}
\hline
\textbf{Algorithm} & \textbf{Hyperparameter} & \textbf{Data Sample} & \textbf{Outcomes Sample} & \textbf{Conclusion}  \\
\hline
 DBSCAN & $\begin{matrix} \epsilon = 3 \times  10^{-4} \\ minPts=5 \end{matrix} $ & Non-scaled & $\begin{matrix}P=0.83 \\ C=0.92 \\ R= 0.99 \\ U=6 \end{matrix}$ &  $\begin{matrix} \text{Worked in cluster detection} \\ \text{ and recovery-rate.} \end{matrix}$  \\ \hline
 
 HDBSCAN & - & - & - & Not good in cluster detection. \\  \hline
 
 DPC & $\begin{matrix} \rho = 8.5 \times 10^{-4} \\ \delta = 0.9986 \end{matrix} $  & Non-scaled & - & 100\% in cluster center detection \\  \hline
   
  DBSCAN & $\begin{matrix} \epsilon = 2.6 \times 10^{-2} \\ minPts=5 \end{matrix} $ & Scaled & $\begin{matrix}P=0.88 \\ C=0.88 \\ R= 0.97 \\ U=7 \end{matrix}$ & Good in cluster detection  \\ \hline
 
 OPTICS & - & Scaled & - & Good in cluster detection. \\  \hline
\end{tabular}}
\caption{Results on SDSS Real Space Galaxy Catalogue sample.}
\label{table:real}
\end{table}

\subsection{Impact of Standardization (results on scaled data)}

While the spatial coordinates were initially defined on a consistent physical, to ensure that our density metrics remained isotropic and independent of the varying scales of the coordinate axes, a standardization protocol (Z-score normalization) was implemented, then OPTICS and DBSCAN were applied again.

By transforming the spatial features to have a mean of zero and unit variance—calculated as 
$$z = \frac{x - \mu} {\sigma}$$

 the numerical bias inherent in raw coordinate ranges is eliminated. 

This preprocessing step yielded a measurable increase in cluster detection sensitivity across the 2dFGRS, SDSS, and Real-Space Galaxy catalogues in OPTICS and DBSCAN algorithms as we can see in tables \label{table:2df}, \label{table:sdss} and \label{table:real}. This confirms that enforcing numerical isotropy is essential for correctly identifying groups in the all geometries.

\section{Results on two-point correlation function (2pcf) on 2dFGRS sample}\label{2pcf2}

A python notebook based on the dataset shown in section \ref{data:2d} is created. A representative sample from the 2dFGRS catalog was extracted to analyze its spatial distribution. By contrasting the empirical 2dFGRS data with a synthetic Poisson distribution, this approach may convey information about the geometry of the matter across the Universe.

	\begin{figure}[h]
	\centering
	\includegraphics[width=0.8\textwidth]{./figs/estimators1.jpg}
	\caption{Natural and Hamilton estimators} on 2dFGRS sample.
	\label{fig:estimators1}
	\end{figure}
	
One important point is the use of scipy.spatial.KDTree package, which allow to improve the time response and calculations by providing an index in a set of k-dimensional space. This indexing strategy was pivotal for the 2PCF pair-counting, reducing the computational complexity from quadratic to logarithmic scales. This ensured that our hyperparameter grid search remained performant even when processing 3D comoving coordinates across $400 h^{-1} Mp$ scales.

The estimators presented in \ref{2pcf1} were successfully applied to this sample and the obtained results show a peak of density over the $100h^{-1} Mp$ (see figures \label{fig:estimators1} and \label{fig:estimators2}). Such peaks can be explained by the Baryonic Acoustic Oscillations (BAO), providing a definitive evidence that this methodology recovers the fundamental 'fingerprint' of the early Universe still visible in the galaxy distribution and consistent with  predictions of the $\lambda-CDM$ model.

While a comprehensive discussion of Baryon Acoustic Oscillations (BAO) lies beyond the scope of this analysis, they provide further empirical evidence for the existence of dark matter. By establishing the initial density fluctuations in the early Universe, dark matter acted as a gravitational scaffold, driving the formation of the virialized halos where the galaxy clusters identified in this study reside.

Finally, these results show galaxies are not sparsed at random positions across the Universe, instead they lie in associated groups and clusters identified by sDBSCAN and sOPTICS. They are the physical manifestations gravitational wells of Dark Matter halos, whose distribution was dictated by the sound horizon of the early Universe."

\begin{figure}[h]
\centering
\includegraphics[width=0.8\textwidth]{./figs/estimators2.jpg}
\caption{ David \& Peebels and Landy \& Szalay estimators} for 2dFGRS sample.
\label{fig:estimators2}
\end{figure}

\section{Conclusions}\label{conclusions}

The results of the comparative analysis along different samples and density analysis presented in this study can be summarized in the following key findings:

\begin{enumerate}
	
	\item Elbow method: It was observed that traditional geometric optimizations, such as the Elbow Method, are unsuitable for clustering detection. As it was said, our results confirm that local density reachability is a more physically accurate proxy for gravitational binding than global variance minimization.
	
%SS	\item Primacy of physical ground truth: statistical heuristics such as Elbow Method do not match with the morphology of the cosmic web. Adopting the Halo-based group distribution as a physical ground truth, we achieved a more robust recovery in DBSCAN and OPTICS.
	
	\item \textbf{OPTICS and DBSCAN} can effectively recover the galaxy groups with reasonable Purity ($\mathcal{P}$) and Completeness ($\mathcal{C}$) but theu yield to low values in Recovery ($\mathcal{R}$).These results are consistent with findings in the literature, such us \cite{Hai-Xia-Ma:2025} and can be attributed to fundamental observational limitations. Specifically, inherent survey challenges and the Redshift-Space Distortions (RSD) detailed in Section \ref{challenges} create structural biases that density-based algorithms alone cannot fully mitigate.
	
	\item \textbf{Better performance} of algorithms based on modified distances along the Line of Sight (SLOS) such as sOPTICS and sDBSCAN: among the tested frameworks, sDBSCAN emerged as the most robust model for recovering the underlying halo distribution. It achieved the highest valid match Ratios in $\mathcal{P}$, $\mathcal{C}$ and $\mathcal{R}$, these results confirm the obtained in \cite{Hai-Xia-Ma:2025}.
	
	\item \textbf{Slightly Superiority of Standardized Data}: The application of Z-score normalization is not an influential factor in improving model performance for OPTICS and DBSCAN. This is true for all metrics such as ($\mathcal{P}$, $\mathcal{C}$, $\mathcal{R}$, and $\mathcal{U}$).
	
	\item \textbf{Re-Real space}: The transition from redshift space to Re-Real space proved essential for accurate structure identification. By correcting for Kaiser and Finger-of-God (FoG) distortions, we significantly reduced the "smearing" of clusters along the line of sight. This correction directly led to higher Purity ($\mathcal{P}$) and Completeness ($\mathcal{C}$) across all evaluated density-based algorithms.
	
	\item \textbf{The Two-Point Correlation Function(2PCF)}: served as a vital statistical validator. The departure of the 2dFGRS sample from the synthetic Poisson baseline provided a quantitative measure of the "crowding" or clustering signal, confirming that our identified groups reside within the high-probability density peaks predicted by $\Lambda$CDM cosmology.
	
\end{enumerate}

\section{Future works}\label{future}
	All methodologies developed in this study provide a foundation for several promising avenues of research:
\begin{enumerate}	
	\item \textbf{Next-Generation Surveys}: While this study utilized the 2dFGRS and SDSS (DR7) catalogs, the computational efficiency of the Density Peak Clustering (DPC) and HDBSCAN frameworks makes them ideal candidates for larger, higher-redshift datasets. Applying these algorithms to the Dark Energy Spectroscopic Instrument (DESI) or the Euclid mission data will allow for a more precise mapping of the cosmic web across cosmic time.
	
	\item \textbf{Machine Learning (ML) Neural Networks (NN)}: Future iterations could move beyond traditional clustering by with Neural Networks(NNs). The Graph ones (GNNs) can be used by treating galaxies as nodes in a graph and their gravitational bonds as edges, we can potentially automate the "Re-Real" space correction process.
	
	\item \textbf{Multi-Wavelength Data}: Another  significant next step would be the cross-correlation of our density-based clusters with X-ray (eROSITA) or Sunyaev-Zeldovich (SZ) effect maps.
\end{enumerate}	
%\section{Results}

%-

%\section{Conclusions and future work}

%-

\phantomsection
\pagestyle{fancy}
\chapter{Glossary}
%\section{Glossary}
%-
\begin{description}
    \item[BAO (Baryon Acoustic Oscillations):] Periodic fluctuations in the density of the visible baryonic matter of the universe, acting as a "standard ruler" for cosmological distances.
	
	\item[CasJobs SQL interface:] Online batch-processing system that gives users access to the multi-terabyte SDSS catalog. Unlike simple web forms that allow for small data downloads, CasJobs is designed for Large-scale Data Mining.
	
	\item[$\Lambda$CDM Model:] The current standard model of cosmology, representing a Universe dominated by Dark Energy ($\Lambda$) and Cold Dark Matter (CDM).
	
	\item[Dark Matter Halo:] A quasi-equilibrium state of dark matter particles. The gravitational "wells" where galaxies and galaxy clusters form and reside. See the introduction section in \cite{Hai-Xia-Ma:2025}.
	
	\item[Kaiser effect:] distorsion in apparent clustering of galaxies that appeaers caused by motions of galaxies as they fall into large structures, causing a "squishing" or flattening along the line of sight. See \cite{Shi:2016} for more details.
	
	\item[$k$-d Tree (k-dimensional Tree):] A space-partitioning data structure used to organize points in a $k$-dimensional space. It allows for high-speed "nearest neighbor" searches, essential for large datasets.
	
	\item[Poisson Distribution:] A random spatial distribution where points are placed independently. This serves as the "null hypothesis" against which the 2dFGRS clustering is measured.
	
	\item[Redshift:] Increase in the wavelength of radiation - tipically ligth-. The redshift takes place for several reasons, one of then is when the source of ligth is further away, for example in an expanding Universe, then they speak about cosmic-redshift.
	
	\item[Redshift Space Distortions (RSD):] An effect where the observed distance of a galaxy is distorted by its peculiar velocity (movement due to gravity), causing clusters to look elongated ("Fingers of God") \cite{Shi:2016}.

	\item[Z-score Normalization (Standardization):] A preprocessing step that transforms data to have a mean of 0 and a standard deviation of 1, preventing one feature (like survey depth) from dominating the distance calculation.
	
\end{description}

\phantomsection
\pagestyle{fancy}
% \chapter{Bibliography}

\addcontentsline{toc}{chapter}{Bibliography}
\bibliographystyle{plain}
\bibliography{refs}

%Appendices section commented for now
% \section{Appendices}

\end{document}