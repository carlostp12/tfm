\pagenumbering{roman}
\setcounter{page}{1}
\pagestyle{plain}
\singlespacing

%%%%%%%%%%%%%%%%
%%% CREDITS %%%
%%%%%%%%%%%%%%%%
\chapter*{Copyright}

\vspace{1cm}

\begin{figure}[ht]
\centering
\includegraphics[scale=1]{images/license.png}
\end{figure}

Copyright © 2025, Carlos Toro Peñas.
Attribution-NonCommercial-NoDerivs 3.0 Spain (CC BY-NC-ND 3.0 ES).

\href{https://creativecommons.org/licenses/by-nc-nd/3.0/es/}{3.0 Spain of CreativeCommons}.

%%%%%%%%%%%%%
%%% RECORD %%%
%%%%%%%%%%%%%
\chapter*{FINAL PROJECT RECORD}

\begin{table}[ht]
\centering{}
\renewcommand{\arraystretch}{2}
\begin{tabular}{r | l}
\hline
Title of the project: & \titulo \\
\hline
Author's name: & \autor \\
\hline
Collaborating teacher's name: & \supervisor \\
\hline
PRA's name: & \profesor\\
\hline
Delivery date (dd/mm/yyyy): & 12/28/2025\\
\hline
Degree or program: & Master's degree in Data Sicience\\
\hline
Final Project area: & \area \\
\hline
Language of the project: & English\\
\hline
Keywords: & \keywords\\
\hline
\end{tabular}
\end{table}

%%%%%%%%%%%%%%%%%%%
%%% DEDICATION %%%
%%%%%%%%%%%%%%%%%%%
\chapter*{Dedication/Quote}


%%%%%%%%%%%%%%%%%%%
%%% Acknowledgements %%%
%%%%%%%%%%%%%%%%%%%
\chapter*{Acknowledgements}

%If deemed appropriate, mention the people, companies or institutions that have contributed to the realization of this project.

%%%%%%%%%%%%%%%%
%%% ABSTRACT %%%
%%%%%%%%%%%%%%%%
\chapter*{Abstract}
\addcontentsline{toc}{chapter}{Abstract}

\onehalfspacing

This work primary focuses on applying density-based algorithms to datasets from major surveys, including the Two-degree Field Galaxy Redshift Survey (2dFGRS) and the Sloan Digital Sky Survey (SDSS. The application will be followed by hyperparameter tuning and a performance assessment to identify the algorithms' strengths and weaknesses in actual galactic group detection. Furthermore, the study integrates a statistical perspective through the Two-Point Correlation Function (2PCF), providing a global measure of galaxy clustering to complement the discrete group detections. These methods establish a robust framework intended for application to next-generation surveys and diverse celestial regions.

\vspace{1.5cm}

\textbf{Keywords}:\keywords.

%%%%%%%%%%%%%%%%
%%% RESUMEN %%%
%%%%%%%%%%%%%%%%
\chapter*{Resumen}
\addcontentsline{toc}{chapter}{Resumen}

\onehalfspacing

Este trabajo tiene como tema central la aplicación de diferentes algoritmos basados en densidad a juegos de datos obtenidos de los cartografiados Two-degree Field Galaxy Redshift Survey (2dFGRS) y Sloan Digital Sky Survey (SDSS). Como resultado de esa aplicación, se hará un ajuste de híper-parámetros así como una evaluación del desempeño de tales algoritmos para analizar sus fortalezas y debilidades en su habilidad para la detección de cúmulos galácticos catalogados. Además se ha querido aportar un punto de vista estadístico mediante la Función de Correlación a Dos Puntos. Se establece mediante este estudio un marco de trabajo para futuras aplicaciones sobre la nueva generación de cartografiados y otras regiones del firmamento.

\vspace{1.5cm}

\textbf{Palabras clave}: \keywordses.