\phantomsection
\pagestyle{fancy}

\chapter{State of the art}
\onehalfspacing
%\addcontentsline{toc}{chapter}{State_of_art}
%\section{Introduction}

This chapter aims to update the reader on the current state of the research area addressed by this work. For this, we will focus on two different parts, first inherent challenges of the data collected by the surveys, and second, a will brief introduction to Machine Learning, whose techniques we will apply in this work.

\section{Surveys}

\section{Machine Learning applyied to cosmology}

We will, before presentig the results, give a brief description of the algorithms employed in this work.

\subsection{Supervised Learning}

Supervised learning focus on find patterns and relations within labeled data. The aim of Supervised Methods is to obtain some knowledge learned from given labeled-data in order to make predictions on some of the classes for new data. In other words given a set of data \( Z = (X, Y) = (X_1,.., X_n, Y_1, ...Y_m) \), we want to find a function F that holds \(Y=F(X) \).

A subset is taken from the original dataset, the so called training data \( Z_i = (X_i, Y_i) \). And then the problem is reduced to find the minumum of a loss function, which measures the difference between \( Y_i \) and \( F(X_i) \) 
The input of any supervised algorithms are some variables 
\subsection{Unsupervised methods}

Unsupervised learning focus on analysis and model of data without using any tags or output classes

\subsection{OPTICS}

\subsection{DBSCAN}

\subsection{HDBSCAN}

\subsection{Non density-based algorithims}

