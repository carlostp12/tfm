\numberwithin{section}{chapter}
\phantomsection
\pagestyle{fancy}

\chapter{Introduction}
\onehalfspacing
%\addcontentsline{toc}{chapter}{Introduction}
%\section{Introduction}
\section{Context and motivation}

When studying the Universe at medium and large scales, we enter the field of galaxy surveys, which rely on dedicated telescopes to obtain large catalogs of galaxies. One objective of these studies is to map vast areas of the Universe. This work relies on data coming from two of these surveys: the Two-degree Field Galaxy Redshift Survey (2dFGRS) and the Sloan Digital Sky Survey (SDSS).

The datasets generated by these surveys are highly suitable for analysis through Machine Learning (ML) methods. Specifically, the redshift feature can be interpreted as a measure of distance by applying cosmological models, as will be detailed in Section \ref{sec:State_of_art}.

The matter distribution in space is far from random; instead, galaxy groups represent the next fundamental structure in the Universe above the level of individual galaxies. At an even higher structural level we find clusters, which are more numerous aggregations of galaxies composed of hundreds or thousands of members. These structures are shaped by dark matter into filaments and voids  \cite{Anatole:2024} to form the so-called cosmic web \cite{Eniasto:2014}.

Determining the structure of groups and clusters is therefore crucial for understanding the distribution of matter in the Universe. This is where clustering algorithms come into play. As will be discussed in section \ref{sec:State_of_art}, the density-based algorithms are the most appropriate for this study.

\subsection{Personal motivation}

Given my personal background and strong interest in astrophysics and cosmology, upon entering the field of Data Science, one can find the vast potential for applying the multiple Machine Learning (ML) techniques to these scientific domains. The study of the Universe's large-scale structure, in particular, stands out as a critical area where ML methods can yield substantial scientific advancements.

%% FUTURE INCLUSION OF M33 Image galaxy of our Local Group
%\begin{figure}[h]
%\centering
%\includegraphics[width=0.5\textwidth]{./figs/M33_2.jpg}
%\caption{M33: A galaxy belonging to our local group.}
%\label{fig:sample_figure}
%\end{figure}


\section{Goals}

There are two list of goals we considered to address separatedly:

\subsection {Main goals}
\begin{itemize}
	\item Apply density-based algorithms to galaxy datasets adquired from 2dFGRS and the Sloan Digital Sky Survey SDSS, in order to obtain a validated model that can effectively approximate the observed structure of groups and clusters.
	\item Determine which of the applied algorithms perform best and investigate the reasons for their effectiveness..
	\item Detect potential outliers and patterns.
	\item Use validation methods to obtain a hyper-parameter tunning in  order to optimize galaxy group/cluster detection.
\end{itemize}


\subsection {Secondary goals}
\begin{itemize}
	\item Generate a visualization map of the data used in this study.
	\item Detect methods to improve this study in future works in following areas: Data Enhancement, Algorithmic Refinement\footnote{For exemple: modify the distance in order to mitigate the already known Redshift-Space distortions along the line of sight\cite{Hai-Xia-Ma:2025}.}.
\end{itemize}

% sustentability from now it will be leave out
% \subsection{Sustainability, diversity, and ethical/social challenges}

This section should assess the positive/negative impact of the project in the following dimensions. It is not required to reach a positive impact in any/all dimensions, but it is necessary to consider and discuss whether there is an impact or not from the beginning of the project.

\begin{description}
    \item[Sustainability] In the development of the project or during its entire lifecycle (e.g., deployment, retirement), does the output of this project have an impact on sustainability and/or ecological footprint (energy/resource consumption/savings, waste, pollution, depletion of raw materials)? Is it affected by laws or regulations on this matter? Considering another perspective, does it affect any of the Sustainable Development Goals (SDG) related to these dimensions? If it does not have any impact, either positive or negative, you should explain how you reached this conclusion and justify your answer.
    \item[Ethical behaviour and social responsibility] Is the outcome of the project too technical to have any positive/negative impact in ethical/social aspects? Does it have an impact on laws/regulations (data, privacy, labour, intellectual property, personal security, …)? Does it adhere to the deontological principles of the profession? Does it endanger/improve/worsen any job position? If it does not have any impact, either positive or negative, you should explain how you reached this conclusion and justify your answer.
    \item[Diversity, gender and human rights] Is the result of this project so technical that it has no positive/negative impact in terms of gender, diversity, or human rights? And in any laws/regulations? And in terms of accessibility, disability, ergonomics and/or information security? If it does not have any impact, either positive or negative, you should explain how you reached this conclusion and justify your answer.
\end{description}

\section{Sustainability, diversity, and ethical/social challenges}

Cosmological findings fundamentally change our understanding of humanity's place in the Universe. Discoveries related to dark matter, dark energy, or the vastness of the cosmos can have profound philosophical implications.

\begin{description}
    \item[Sustainability] The most direct social responsibility implication lies in the immense power required to process and store astronomical data. This work while purely theoretical, relies on an infrastructure that carries a heavy sustainability burden. That is why focus on computational efficiency directly translates into lower energy consumption, this is the most tangible sustainability implication in this project.
	
    \item[Ethical behaviour and social responsibility] The major impact on this matter concerns the use of resources. This project mitigates resource impact by using only shared, openly licensed libraries and datasets whose usage terms are fully respected by the authors. Of course, all references to the utilized datasets and other previous works are properly cited, given that this project relies fundamentally upon them.
	
	Communicating the outcomes clearly and accurately is also a commitment from the authors of this work.
	
    \item[Diversity, gender and human rights] Astronomy and cosmology, like many sciences, have historically struggled with issues of diversity and inclusion gender and human rights matters\footnote{An example in gender matter can be seen in the eighth chapter of the documentary television series Cosmos: A Spacetime Odyssey, titled "Sisters of the Sun," hosted by Neil deGrasse Tyson.}, 
	
	The authors are committed to respecting these questions throughout this work. More generally, the further advancement in science benefits society by better equipping it to address issues on these matters.	
\end{description}

To conclude this section, this work uses powerful analytical methods derived from ML techniques, all tools could be adapted for surveillance, military intelligence, or other uses that might infringe on human rights or privacy. Scientists must be mindful of how their methods and code are shared.

\section{Approach and methodology}

We will apply classical phases drawn from the data life-cycle, which cover:

\begin{itemize}
 \item{Collection}: download datasets drawn from surveys such as the SDSS and 2DFGRS to generate galaxy clustering models. These datasets are available at \cite{Blanton:2005}:

https://gax.sjtu.edu.cn/data/Group.html
 
 \item{Storage}: keep donwnloaded data set in csv files.
 \item{Preprocessing}: stage containinig the tasks of cleaning, filtering, sampling, and fusion.
 \item{Analysis stage}: which includes model building through the application of the algorithms and validation of the outcomes.
 \item{Visualization}: graphical view of the results.
\end{itemize}


The Analysis Stage will utilize an iterative methodology dedicated to enhancing the robustness and accuracy of the model outputs. All models employed in this stage will consist of unsupervised algorithms, with a particular focus on density-based clustering techniques to identify structures and outliers within the data.

To evaluate the performance of these models, a traditional criteria will be followed (see section \label{sect:aplication} for more details):

\begin{itemize}
    \item {Detected Clusters}: clusters successfully classified (often referred to as True Positives at the group level).

	\item {Undetected Clusters}: clusters not found or not identified in the output-clusters set (equivalent to True Negatives at the group level).

	\item {Cluster Purity Ratio}: proportion of members in a output-cluster that actually belong to the underlying cluster/group structure.

	\item {Cluster Completeness Ratio}: proportion of members of a true underlying group/cluster that are successfully included within the detected output-cluster.

	\item {Misclassified Members}: individual data points (galaxies) belonging to an actual structure but classified outside of any detected output-cluster (often referred to as False Negatives at the individual member level).

	\item {External Data Classified as Members}: individual data points (galaxies) not belonging to any actual group but erroneously classified inside a detected output-cluster often referred to as False Positives at the individual member level).
\end{itemize}

We utilize both Python and R to execute the various stages of this study. Python handles the data management and statistical estimation of the two-point correlation function. Meanwhile, the implementation and performance assessment of the density-based clustering frameworks are carried out in R, ensuring a robust statistical analysis of the galaxy group distributions.

\section{Schedule}

A Gantt diagram in figure \ref{fig:stages_figure} shows the different stages of the project development. Excluding the final project defense, the stages have been grouped in three blocks:

\begin{figure}[h]
\centering
\includegraphics[width=1.0\textwidth]{./figs/gantt_conogram.png}
\caption{Stages of the project.}
\label{fig:stages_figure}
\end{figure}

\begin{itemize}
	\item Planning (shown in green) involves gathering resources and defining the project's objectives.
	\item Technical development (shown in red) includes design, data processing, method application and outcomes assessment.
	\item Research and writing (shown in blue).
\end{itemize}

An iterative and continuous review of the results will be performed throughout the analysis process due to several causes: issues stemming from the algorithms, data processing, and the workflow itself. As a result, initial objectives might be rearranged and redefined. This is why aditional objetives stage is necessary.
