\phantomsection
\pagestyle{fancy}

\chapter{Implementation}
\onehalfspacing
%\addcontentsline{toc}{chapter}{State_of_art}
%\section{Introduction}


\section{ETL processing of datasets}

A python framework has been implemented in order to extract valid-format (CSV) datasets to serve as input of our analysis. The python preproceesign framework transform the raw files on a final CSV the fields as in the table \ref{table:data}.

The final objective consist in get the association of individual galaxies with their respective dark matter halos or larger groups, enabling us to apply density based algorithm analysis. There are three basic steps to merging:

\begin{enumerate}
	\item Format the galaxy catalog to a CSV file.
	\item Format the group catalog to a CSV file.
	\item Merge galaxy and group catalog and transform coordinates and distances.
\end{enumerate}	

The synthesis results in a unified dataset formatted for computational efficiency. The header of this processed file, which contains the spatial and environmental metadata for each galaxy, is displayed in Figure \ref{fig:cat}.

Due to the varying structures of the survey catalogs, the data acquisition and preparation phase is divided into distinct modules to ensure inter-survey compatibility.

\subsection{2dF Galaxy Redshift Survey (2dfGRS)}


\begin{enumerate}
	\item \textit{2dfGRS.dat}: Which comprises 245,591 individual galaxy entries. To ensure high-fidelity measurements and minimize redshift uncertainty, we applied a quality constraint of $Q \geq 3$, excluding objects with poorly determined spectral features or low signal-to-noise ratios
	
	\item \textit{group\_members}: a supplementary group-membership file consisting of 104,913 galaxies.
\end{enumerate}	


\subsection{Sloan Digital Sky Survey (SDSS)}

Among the diverse datasets provided by the SDSS archive, the imodelC\_1 file was identified as the most suitable for this analysis. It contains the required astrometric and photometric parameters to accurately cross-match with our random catalogs, allowing for a robust calculation of the large-scale structure signal.

\begin{enumerate}
	\item \textit{SDSS7}: Galaxy catalog of the survey.
	\item \textit{imodelC\_1} Comprises 245,591 entries for each galaxy. (again we applied a quality constraint of $Q \geq 3$.)
\end{enumerate}	
	
Our pipeline performs a multi-source integration of the raw data files, executing the necessary joins and quality filters to produce a unified CSV file optimized for clustering analysis. A representative sample of this finalized data product is provided in Figure \ref{fig:cat}, demonstrating the successful synthesis of spatial and environmental metadata.

\begin{table}[]
\centering
\begin{tabular}{|l|l|l|}
\hline
\textbf{Field-Name} & \textbf{FDescription} & \textbf{Field-Type} \\
\hline
 GAL\_ID & ID of galaxy en each catalog & numerical \\ \hline
 ra & Right ascension coordinate  & decimal \\  \hline
 dec & Declination coordinate  & decimal    \\ \hline
 x & X cartesiaan coordinate & decimal    \\ \hline
 y & Y cartesiaan coordinate & decimal    \\ \hline
 z & Z cartesiaan coordinate & decimal    \\ \hline
 redshift & cell8 & decimal \\ \hline
 dist & Raw distance value & decimal  \\ \hline
 GROUP\_ID & id-group galaxy belongs to & numerical \\ \hline
 
\end{tabular}
\caption{Datasheet metadata.}
\label{tab:data}
\end{table}

\subsection{Real Space Galaxy Catalogue}
Finally, we incorporated the SDSS 'Real Space Galaxy Catalogue,' which provides galaxy coordinates reconstructed to account for redshift distortions as we explained in sections \ref{section:slos} and \ref{section:real} . This enables a robust comparison between observed clustering and theoretical predictions by isolating the isotropic real-space correlation function.

	\begin{figure}[h]
	\centering
	\includegraphics[width=0.5\textwidth]{./figs/catalog.jpg}
	\caption{ Final format of datasheet }
	\label{fig:cat}
	\end{figure}
	

\section{The two-point correlation function: another point of view of density analysis}

Drawing on the comparative analysis provided by \ref{Kerscher:2000}, we implement four distinct estimators to quantify the clustering signal. This selection allows for a rigorous cross-examination of the statistical bias and variance inherent in each method when applied to large-scale galaxy catalogs.

Natural
\begin{equation} \label{eq:43}
	\widehat{\xi} _{N}= \frac{DD}{RR} - 1
\end{equation}.

Davids and Peebels:
\begin{equation} \label{eq:41}
	\widehat{\xi} _{DP}= \frac{DD}{DR} - 1
\end{equation}.

Hamilton:
\begin{equation} \label{eq:42}
	\widehat{\xi} _{Ha}= \frac{DD \, RR}{DR^2}
\end{equation}.

Landy and Szalay:
\begin{equation} \label{eq:44}
	\widehat{\xi} _{LS}= \frac{DD -2DR +RR}{RR}
\end{equation}.

Upon applying these estimators to the 2dFGRS dataset, we successfully recovered a distinct clustering feature at approximately $100 \ h^{-1}\text{Mpc}$. This signal is statistically consistent with the predicted scale of Baryon Acoustic Oscillations (BAOs), representing a detection of the primordial acoustic horizon in the late-time galaxy distribution.


\section{Applying density-based algorithms to data}

We have tested several algorithms in order to obtain a model of density clustering.

OPTICS: To crete the reachability plot.
OPTICSXi: Generate clusters in hierarchical mode.
DBSCAN: Generate clusters by reachability plot threshods.
HDBSCAN: Generate cluster form variable density regions.
DPC (Density Peaks Clustering): Generate cluster from density peaks.
OPTICS (on scaled data): The same in OPTICS with scaled data.
DBSCAN  (on scaled data): The same in DBSCAN with scaled data.
sOPTICS and sDBSCAN: Generate reachability plot and clusters with the modified distances as explained at \ref{section:slos}

It is important to emphasize that this study departs from traditional unsupervised clustering objectives, such as minimizing intra-cluster variance via the Elbow Method. Instead, we treat the Halo-based group distribution as a physical ground truth. Consequently, many standard clustering algorithms —and their default hyperparameter configurations— may fail to yield results consistent with our model of virialized galaxy groups, as they are not intrinsically designed to account for the specific density profiles of dark matter halos.

The performance of each algorithm is evaluated based on its Recovery Rate of known halo members. We prioritize models that maximize the completeness and purity of identified groups relative to the 2dFGRS/SDSS group catalogs, rather than those that simply minimize global silhouette scores.

With this in mind and guided by \cite{Hai-Xia-Ma:2025} we defined:
	
\begin{equation} \label{eq:purity}
	Purity = \frac{total\_in\_cluster\_group}{total\_in\_cluster}
\end{equation}.

$total\_in\_cluster$: number of elements in output-cluster.
$total\_in\_cluster\_group$: number of elements from orignal group present in an output-cluster. 
$total\_in\_cluster_group$: number of elements from orignal group present in an output-cluster.

\begin{equation} \label{eq:completeness}
	Completeness = \frac{total\_in\_cluster\_group}{total\_in\_group}
\end{equation}.

Following the same concept as \cite{Hai-Xia-Ma:2025}, an output-cluster is said to be $Pure$ if $Purity \geq \frac{2}{3} $ and complete if $Completeness \geq \frac{1}{2} $.

With this concepts we evaluate the following ratios:
  
\begin{equation} \label{eq:completeness}
	Fp = \frac{number\_pure\_groups}{predicted\_groups}
\end{equation}.

\begin{equation} \label{eq:completeness}
	Fr = \frac{number\_pure\_complete\_groups}{predicted\_groups}
\end{equation}.
Fr only for complete an pure output-clusters:

\begin{equation} \label{eq:completeness}
	Fr = \frac{total\_in\_group}{number\_non\_isolated\_galaxies}
\end{equation}.


\section{Applying 2pcf estimators to 2dFGRS datasheet}

-- SELECT TOP 10 o.objID, o.ra, o.dec, p.z
select count(o.objID) 
FROM photoprimary o INNER JOIN Photoz p ON o.objID = p.objID 
WHERE
p.z BETWEEN 0.2 AND 0.4 AND
--o.ra BETWEEN 30 AND 60 AND
o.ra BETWEEN 165 AND 195 AND
o.dec BETWEEN 0 AND 50 AND
(o.dered_r-o.dered_i) < 2 AND
o.cmodelmag_i-o.extinction_i BETWEEN 17.5 AND 19.9 AND
(o.dered_r-o.dered_i) - (o.dered_g-o.dered_r)/8. > 0.55 AND
o.fiber2mag_i < 21.7 AND o.devrad_i < 20. AND
o.dered_i < 19.86 + 1.60*((o.dered_r-dered_i) - (o.dered_g-dered_r)/8. - 0.80) 
 