\phantomsection
\pagestyle{fancy}

\chapter{Implementation}
\onehalfspacing
%\addcontentsline{toc}{chapter}{State_of_art}
%\section{Introduction}



\section{ETL processing of datasets}

A python framework has been implemented in order to extract valid-format (CSV) datasets to serve as input of our analysis. The python preproceesign framework transform the raw files on a final CSV containing the fields:

\begin{enumerate}
	\item \textit{GAL_ID}: ID of galaxy en each catalog.
	\item \textit{ra}: numerical right ascension coordinate.
	\item \textit{dec}: numerical declination coordinate.
	\item \textit{x}: x cartesiaan coordinate.
	\item \textit{y}: y cartesiaan coordinate.
	\item \textit{z}: z cartesiaan coordinate.
	\item \textit{redshift}: the measured redshift value from the survey.
	\item \textit{dist}:  calculated proper distance.
	\item \textit{GROUP_ID}: id-group galaxy belongs.
\end{enumerate}

Different catalogs have different structures, that is why we splitted in several sections.

\subsection{2dF Galaxy Redshift Survey (2dfGRS)}

We need to download downloaded:

\begin{enumerate}
	\item \textit{2dfGRS.dat}: file which contains 245591 entries for each galaxy. (only those which quality >=3 are selected.)
	\item \textit{group_members}: which is other raw data-file relating 104913 galaxies with their groups.
\end{enumerate}	

The merge of both files gives a total of 104912 rows in the final catalog. This is because not all galaxies belongs to a group. There are 3 basic steps to merging:
	1. format the galaxy catalog to a CSV file.
	2. format the group catalog to a CSV file.
	3. Merge galaxy and group catalog and transform coordinates and distances.
The final result is the 2dfgrs-valid.csv file whose head is showed at \ref{fig:} 

\subsection{Sloan Digital Sky Survey (SDSS)}

The package downloaded contains several files depending on the magnitude type, etc. For our porposes, the imodelC_1 one is perfectly valid.

\begin{enumerate}
	\item \textit{SDSS7): galaxy catalog of the survey, contains .
	\item \textit{imodelC_1} file which contains 245591 entries for each galaxy. (only those which quality >=3 are selected.)
\end{enumerate}	
	
The framework process both files to obtain the final csv file. 	
Also a final dataset was downloaded with the SDSS catalog corrected the redshift distorsion, the so called "Real Space Galaxy Catalogue".




\section{The two-point correlation function: another point of view of density analysis}

Other way to study the density of galaxies in the universe is through the two-points correlation function (2pcf) which depicts a statistical measure to quantify how clusteread is a distribution of galaxies. It works by calculating the probability of finding a two objects at a given distance.

The Fourier transform of the two-point correlation function is the power spectrum, which is often used to describe density fluctuations observed in the cosmic microwave background.

There exists several stimators for 2pcf, according to \ref{Kerscher:2000} we choose four of them:

\begin{equation} \label{eq:4_1}
	\widehat{\xi_{LS}} = \frac{DD}{RR} -1
	\caption{Davids and Peebels}
\end{equation}.

\begin{equation} \label{eq:4_2}
	\widehat{\xi_{LS}} = \frac{DD-DR}{RR}
	\caption{Hamilton}
\end{equation}.

\begin{equation} \label{eq:4_3}
	\widehat{\xi_{LS}} = \frac{DD}{DR} -1
	\caption{Hewett}
\end{equation}.
\begin{equation} \label{eq:4_4}
	\widehat{\xi_{LS}} = \frac{DD-2DR+RR}{RR}
	\caption{Landy and Szlay}
\end{equation}.


